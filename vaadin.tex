	\chapter {Vaadin}
	\label{ch:vaadin}
	
	 ``Vaadin Framework is a Java web application development framework that is
	designed to make creation and maintenance of high quality web-based user interfaces easy.
	 Vaadin supports two different programming models: server-side and client-side. 
	 `` \cite[pr1.1]{bookVaaidn}
	 \emph{Client-side} Vaadin code is executed in the web browser as JavaScript
	 code.
	 \emph{Server-side} code is executed on the server as Java code on the Java
	 Virtual machine(JVM).
	 
	 Client-side code is originally written in Java and then
	 compiled to JavaScript using \emph{Vaadin Client Compiler}. Vaadin Client Compiler is based on Google
	 Web Toolkt(GWT) which provides the opportunity to write code in Java and
	 execute it in the browser. The client-side code is responsible for rendering
	 the user interface and send user interaction to the server.
	
	 Nowdays there are a lot of standarts and recommendations for web
	 developers published by World Wide Web Consortium (W3C) or  International
	 Organization for Standardization (ISO), including recommendations for
	 markup languages (HTML,XML,SVG), Document Object Models and standarts
	 for JavaScript. Inspite of all the standarts the difference between browsers
	 and versions might be significant for the developer. The differences may vary
	 from supporting/not supporting different CSS tags and HTML5 features,
	 different event handling and simply bugs. Vaadin client Compiler and GWT provide a wide browser
	 support, eliminating the difference between browsers, and helping a
	 developer to concentrate on essential parts of the application, instead of
	 wasting time on cross-browser support. Vaadin uses screenshot comparesment
	 (which is mentioned in part ?) as a part of regression testing, which brings
	 confindence to the developer that Vaadin components will not change their
	 appearence unexpedly with changing the version of Vaadin. 
	 
	 A server-side code runs as a servlet in a Java web server, serving HTTP
	 requests.
	 The VaadinServlet is normally used as the servlet class. The servlet receives client requests
	 and inteprets them as events for a particular user session.
	 Events are associated with user interface components and delivered to the event listeners defined in the application.
	 If the UI logic makes changes to the server-side user interface components, 
	 the servlet renders them in the web browser by generating a response.
	 
	 
	 As mentioned before both client-side and server-side code in Vaadin is written
	 in Java. This positively influences the development process in the following
	 way:
	  1. The developer does not need to know several programming languages and one
	  person may be involved in the developing of both front-end and back-end. This
	  might be an important factor for small teams and speeds up the develpment
	  process.
	  2. Vaadin brings the power of Java into the web development. Due to
	  TIOBE index \cite{tiobeIndex} Java and C are the two most popular programming
	  languages since 2002. This fact allows developer to use a great amount of
	  already-made solutions such as building tools Maven \cite{maven}, Ant
	  \cite{ant}, testing tools \cite{junit}, frameworks as akka \cite{akka} and
	  other libraries like yodatime \cite{yodatime}, guava \cite{guava} for different stages of development
	  process.
	  
	\subsection {Intergrating Vaadin with other frameworks}
	  As mentioned in (link to scratch) one of the key factors for a successful
	  development process is to pick a right toolset. Integrating server-side
	  code and other Java frameworks or libraries is easy and needs the minimum
	  amount of ``glue code''. In the following section we will show how to
	  integrate Vaadin and Akka.
	   
	  \subsection {Integrating witk Akka}
	  Akka is a Java/Scala framework for writting concurrent, fault-tolerant and
	  scalable applications. The Akka framework was adopted by many organizations
	  in a big range of industries all from investment and merchant banking, retail and social media,
	   simulation, gaming and betting, automobile and traffic systems, health care,
	   data analytics and much more\cite {akkakUseCases}. Using akka with Vaadin
	   gives an opportunity to develop concurrent, high responsive web
	   applications. Using Akka provides an
	   opportunity of developing multithreaded easily-scalable backend, when
	   changes are pushed to the client-side asynchronously(not-blocking UI). 
	   We created a sample project which solves the classical Producer-Consumer
	   problem. In this particular example Prouducer is a pool of subscriptions and
	   Consumer is a service provide, which service should be granted, according to
	   business logic.
	   
	   The main idea is the following:
		 We have one service provider, which approves requests, a request in our case
		 is a subscription for some abstract service.
		 We request several subscriptions from UI, each subscription is a thread safe task. The provider can work simulteneosly
		 only with fixed amount of tasks, if the limis is exceeded the provider sends a message to a subscription,
		 that the subscription request is put into the queue and will be initiated a
		 timeout. The information about approved subscriptions is send to the UI
		 asynchronously. The idea in this example is that when the Provider is
		 overloaded and has no resources to process the request it sends the message
		 to the client, that its request was recieved and will be processed later. The
		 client recieves this information instead of just handling. So the goal to get
		 a higly responsive application is arhcieved. The complete example can be
		 downloaded here \cite{vaadinAkka}
	
	  As we see in the above example Vaadin gives an opportunity to use already
	  made Java solutions with minimum/no overhead. The server-side code is pure
	  Java and can use a third-party framework or third-party library to ease the
	  development process. For testing server-side code we can use JUnit framework
	  which already has become a standart in the Java world.
	  
	  The difficulties come when testing client side. The client side code is
	  compiled to Javascript and executed in a web-browser, that is why we can not
	  use a Junit framework to test it. The challenge is that we need to have a
	  framework which can operate on a html page raising javascript events and
	  simulating user interaction, but also send requests related to these
	  interactions to the server side. In the following chapter we will
	  present several ideas about such testing framework.	   