	
\appendix
\section{Appendix A}
\label{appendixA}

\lstset{style=console}
\begin{lstlisting} [caption=Generated C\&R test  example using Selenium IDE ,label={lst:capturereplay}]
<tr> 
  <td>open</td>
  <td>
    /tutorials/selenium/selenium_record_replay.htm
  </td>
  <td>type</td>
  <td>id=number1</td>
  <td>123</td>
</tr>
<tr>
  <td>type</td>
  <td>id=number2</td>
  <td>123</td>
</tr>
<tr>
  <td>click</td>
  <td>id=add</td>
  <td></td>
</tr>
<tr>
  <td>verifyValue</td>
  <td>id=total</td>
  <td>246</td>
</tr>
\end{lstlisting}
	 
\lstset{style=a1listing}
\begin{lstlisting} [caption=Test UI class example,label={lst:appTestUI}]
@Theme("mytheme")
@Widgetset("com.example.testbench.MyAppWidgetset")
public class MyUI extends UI {
	@Override
	protected void init(VaadinRequest vaadinRequest) {
	final VerticalLayout layout = new VerticalLayout();
	layout.setMargin(true);
	setContent(layout);

	Button button = new Button("Click Me");
	button.addClickListener(new Button.ClickListener() {
	@Override
	public void buttonClick(ClickEvent event) 
		layout.addComponent(new Label("Thank you for
			clicking")); }
	});
	layout.addComponent(button);

	}
    
	@WebServlet(urlPatterns = "/*", name = "MyUIServlet",
		asyncSupported = true)
	@VaadinServletConfiguration(ui = MyUI.class,
		productionMode = false)
	public static class MyUIServlet extends VaadinServlet {
	}
}
\end{lstlisting}

\lstset{style=a1listing}
\begin{lstlisting} [caption=Test Bench class example,label={lst:appTestClass}]
public class ButtonTest extends TestBenchTestCase {
	public static final String baseUrl =
		"http://localhost:8080";
    
	@Before
	public void setUp() throws Exception {
		// Set WebDriver
	setDriver(new FirefoxDriver());
}

	@After
	public void tearDown() throws Exception {
		getDriver().quit();
	}
    
	@Test
	public void testClick() {
	//Open URL
	getDriver().get(baseUrl + "?restartApplication");
	ButtonElement button = $(ButtonElement.class).first();
		button.click();
		LabelElement label = $(LabelElement.class).first();
		String text = label.getText();
		Assert.assertEquals("Thank you for clicking", text);
	}
    }
\end{lstlisting}


  	\lstset{style=a1listing}
  	\begin{lstlisting} [caption=Selenium test for Vaadin application,label={lst:seleniumVaadin}]
public class AppTest extends TestCase{
  WebDriver driver;
  String UIUrl, nodeURL;	
	@Override @Before
    public void setUp() {
      UIUrl = "http://demo.vaadin.com/dashboard/";
      driver = new FirefoxDriver();
    }

  @Test
  public void testWithoutTestbench() {
    driver.get(UIUrl);
    driver.manage().timeouts().
      implicitlyWait(5, TimeUnit.SECONDS);
    List<WebElement> elements = driver.findElements(By
      .className("v-button"));
    if (elements.isEmpty()) {
      throw new RuntimeException("No buttons found");
    }
    elements.get(0).click();
    driver.findElement(By.id("dashboard-edit")).click();
    WebElement searchField =
     driver.findElements(By.className("v-textfield")).get(0);
    
    searchField.clear();
    searchField.sendKeys("New Dashboard");
    searchField.sendKeys(Keys.TAB);
    WebElement searchButton = findButtonByCaption("Save");
    searchButton.click();
    driver.manage().timeouts().
      implicitlyWait(5, TimeUnit.SECONDS);
    String title = driver.findElement(By.
      id("dashboard-title")).getText();
    Assert.assertEquals("New Dashboard", title);
    }

  public WebElement findButtonByCaption(String caption) {
    List<WebElement> buttons = driver
      .findElements(By.className("v-button"));
    for (WebElement button : buttons) {
      if (button.getText().equals(caption)) {
        return button;
      }
    }
    return null;
    }

  public WebElement findButtonByCaption(
    WebElement parent, String caption) {
   
    List<WebElement> buttons = parent
      .findElements(By.className("v-button"));
    
    for (WebElement button : buttons) {
      if (button.getText().equals(caption)) {
        return button;
      }

    }
    return null;
  }

 @After
 public void afterTest() {
  driver.quit();
 }
}
\end{lstlisting}

	
	\ref{lst:testbenchVaadin}.
	\lstset{style=a1listing}
  	\begin{lstlisting} [caption=TestBench test,label={lst:testbenchVaadin}]
public class TestBenchTest extends TestBenchTestCase {

    WebDriver driver;
    String UIUrl, nodeURL;

    @Before
    public void setUp() throws Exception {
        UIUrl = "http://demo.vaadin.com/dashboard/";
        setDriver(new FirefoxDriver());
    }

    @Test
    public void test1() {
        getDriver().get(UIUrl);
        $(ButtonElement.class).first().click();
        $(ButtonElement.class).id("dashboard-edit").click();
        TextFieldElement searchField =
           $(TextFieldElement.class).first();
        searchField.setValue("New dashboard");
        $(ButtonElement.class).caption("Save").first().click();

        String title = $(LabelElement.class).
          id("dashboard-title").getText();
        Assert.assertEquals("New dashboard", title);
    }

    @After
    public void afterTest() {
        // driver.quit();
    }
}  	
\end{lstlisting}

 \lstset{style=a1listing}
\begin{lstlisting} [caption=JBehave test example,label={lst:jbehaveTest}]
public class FilterTableSteps extends TestBenchTestCase {
	TableTestUI page;
	@BeforeScenario
		//open web page
	public void beforeScenario() {
		setDriver(TestBench.createDriver(
			new FirefoxDriver()));
		getDriver().get("http://localhost:8080");
	}
	
	@AfterScenario
	public void afterScenario() {
		getDriver().quit();
	}
	
	@Given("web-page with table")
		public void theFrontPage() throws Throwable {
		page = PageFactory.initElements(
			getDriver(), TableTestUI.class);
	}
	
	@When("typing $value to filter field")
	public void filterTable(String value)
		throws Throwable {
		TableElement table =
			page.$(TableElement.class).first();
		TextFieldElement filterElement=
			table.$(TextFieldElement.class).
			id("filter-field").first();
		filterElement.setValue("special");
	}
	
	
	@Then("value in row $rowNumber 
		and cell $cellNumber is $expectedValue")
	public void checkValueInCell(int rowNumber,
		int cellNumber,String expectedValue) throws Throwable {
		TableElement table = page.$(TableElement.class).first();
		TableRowElement row=table.getRow(rowNumber);
		assertEquals(row.getCell(cellNumber).
			getValue(),expectedValue);
	}
}
\end{lstlisting}

\lstset{style=a1listing}
\begin{lstlisting} [caption=JBehave configuration example,label={lst:jbehaveConfig}]
public class SimpleConfig extends JUnitStory {
	//Specify the configuration, starting from default
	//MostUsefulConfiguration, // and changing only what is needed
    
	@Override
	public Configuration configuration() {
		return new MostUsefulConfiguration()
		// where to find the stories
		.useStoryLoader(new LoadFromClasspath(this.getClass())) 
		// CONSOLE and TXT reporting
		.useStoryReporterBuilder(new StoryReporterBuilder()
			.withDefaultFormats()
			.withFormats(Format.CONSOLE, Format.TXT));
	}
 
	// Specify the steps classes
	@Override
	public InjectableStepsFactory stepsFactory() {    	
	// varargs, can have more than one step class
		return new InstanceStepsFactory(configuration(),
		new FilterTableSteps());
	}
}
\end{lstlisting}
