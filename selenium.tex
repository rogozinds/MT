	\chapter {Selenium}
	\label{ch:selenium}
      Selenium is a set of different software tools each with different
      approaches to supporting test automation. The entire suite of tools allows
      many options for locating UI elements and comparing expected test results
       against actual application behavior \cite{seleniumSite}.
       Selenium provides implementation both of C\&R \ref{sec:captureReplay}
       \ref{sec:programTests} and programmable tests models.  
       
       \textbf{Selenium IDE} - is a development environment with graphical
       interface for building test scripts.
		Selenium IDE has a recording feature,  which records user actions as they are
		performed and then exports them as a reusable script in one of many programming languages that can be later executed.
		An example of recorded test \ref{lst:capturereplay} opens a URL type two
		numbers to fields with ``number1'' and ``number2'', then -
		clicks and ``add'' button and verify the result see Appendix \ref{appendixA}.
		
       \textbf{Selenium  WebDriver} makes direct calls to the browser using
       browser's native support for automation. WebDriver provides an
       API for simulating user actions in a Web browser, for example
       clicking a mouse button or typing on a keyboard.
       WebDriver provides a multi-browser support and allows same tests to be executed in different
       environments see figure~\ref{fig:webdriver}.
       
	  \begin{figure}
	  \centering
  		\includegraphics[width=0.75\textwidth]{webdriver_structure}
  		\caption{Web structure}
  		\label{fig:webdriver}
		\end{figure}
		
    	\textbf{Selenium Remote Control (RC)} is an old version of Selenium
    	WebDriver, currently supported only in maintenance mode.
       
       \textbf{Selenium Grid} allows to execute tests on different machines.
       Selenium Grid is useful for projects with large amount of tests or test
       suites that must be run in multiple environments.
       
		Grid allows to add several physical machines to a test cluster. Grid uses the
		term ``hub'' for a central point where all tests are loaded. Hub is responsible
		for distributing the tests across nodes. ``Node'' is  a remote machine with
		specific configuration which is attached to the hub see
		figure~\ref{fig:selnium_grid}. Nodes are totally separated from each other and
		may have different operating systems and browsers. Hub ``decides''
		for each test suite on which node it should be executed based on tests configuration.
		By default every node starts eleven browsers : five Firefox, five Chrome and
		one Internet Explorer.
		
		\begin{figure}
		\centering
  		\includegraphics[width=0.75\textwidth, center]{selenium_grid}
  		\caption{Selenium Grid structure}
		\label{fig:selnium_grid}
		\end{figure}
		
		Selenium Grid is released as a separate jar file, so to setup a test hub you
		only need to have JRE(Java Runtime Envirenment) installed. For starting hub
		run selenium-server with hub parameter see \ref{lst:starthub}. For starting
		a new node specify a ``webdriver'' parameter and the URL of the hub running
		\ref{lst:startnode}.
		
		\lstset{style=console}
\begin{lstlisting} [ label={lst:starthub},language=bash, caption=Start hub ]
java -jar	selenium-server-standalone-2.30.0.jar -role hub
\end{lstlisting}

    \lstset{style=console}
\begin{lstlisting} [caption=Start node, label={lst:startnode},language=bash]
java -jar selenium-server-standalone-2.30.0.jar 
-role webdriver
-hub http://http://192.168.1.1:4444/grid/register
-port 5566
\end{lstlisting}
		

	After you have set up the basic configuration of Selenium Grid you can open a
	tested Web page ``driver.get(UIUrl)'' and use ``Find Element" or ``Find
	Elements" to get Web elements on your Web page see example
	\ref{lst:starthub2}in the Appendix A. 
	 Selenium  supports searching elements by id, tag, class, Xpath.
	 For example to find an element with a class ``profile'' you need to use
	 By.className query object see example ~\ref{lst:classSearch}. There is no
	  ideal strategy for searching the required element, a developer
	decides which one to choose based on requirements.
	
	\lstset{style=a1listing}
	\begin{lstlisting} [caption=Search element by class,label={lst:classSearch}]
WebElement avatarElement = 
	 driver.findElement(By.className("profile"));
	\end{lstlisting}