	
	 \chapter{Introduction}
	 \label{ch:intro} 		
	 Web applications provide critical services to our society,
	 ranging from the financial and commercial sector,to the public
	 administration and health care.The wide spread use of web
	 applications as the natural interface between a service and its
	 users puts a serious demand on the quality levels that web
	 application developers are expected to deliver. At the same time,
	 web applications tend to evolve quickly, especially for what
	 concerns the presentation and interaction layer. The release
	 cycle of web applications is very short, which makes it difficult
	 to accommodate quality assurance (e.g.,testing) activities in 
	 the development process when a new release is delivered. For
	 these reasons, the possibility to increase the effectiveness and
	 efficiency of web testing has become a major need and several 
	 methodologies, tools and techniques have been developed over time.
	 
	 In these paper I will present Vaadin framework and testing tool for Vaadin,
	 called Vaadin Testbench.
	
	``Vaadin Framework is a Java web application development framework that is
	designed to make creation and maintenance of high quality web-based user interfaces easy.
	 Vaadin supports two different programming models: server-side and client-side. 
	 `` \cite[pr1.1]{bookVaaidn}
	 Vaadin TestBench is a tool for automated user interface testing of web
	 applications on all platforms and browsers. \cite{vaadinTestbenchSite}
	 
	 In the Fall 2014 I was a part of the team which developed Vaadin Testbench
	 4.0.0 and released it in the beginging of December. Web testing tools is a
	 new topic and in the this work I will represent the main ideas and challenges
	 of web testing and how they were solved during Testbench development. I will
	 also describe Vaadin framework, because Testbench is focused on testing web
	 applications written with Vaadin. I will also describe the working flow, what
	 tools and methodologies were used and how the final product might help Vaadin
	 developers.
	 
	 The goal of this work is to provide a tool for a developer, that will help to
	 write tests which can simulate user actions on the web page. The main
	 challenge is that code written with Vaadin might execute both on the
	 client-side and server-side, so that events happen on the client-side will be
	 properly sent to the server-side. Another challenge is to develop an universal
	 easy to use testing tool for Vaadin framework with a clear API.
	 
	  The result of the work was Testbench 4.0.0 released in December 2014.
	  Several user tests have shown, that a person with experience in Java and
	  Vaadin, but without any experience using Testbench, needs 15 minutes to setup
	  the envirenment and write a simple ``button-click'' test. We consider this
	  result as a success. Vaadin is an open-source project and Testbench is
	  available with free for non-comercial use license. So every person from Vaadin community can try
	  Testbench and take a look on results of our work and decide does it suits
	  his/her own needs.
	  
	  This document is structured as follows. Chapter~\ref{ch:style}
	  discusses briefly the basics of writing and presentation style
	  regarding the text, figures, tables and mathematical
	  notations. Chapters~\ref{sec:ref_styles} and \ref{ch:concl} summarize
	   the referencing basics and the whole document. There are two example
	appendices as well (Appendix A and B).
% ~\ref{app:A} and \ref{app:B}). Labels do not with \chapter*