	
	 \chapter{Introduction}
	 \label{ch:intro} 		
	The 21st century has become an era of Web applications. A software system
	developed as a Web based application allows an end user to access data via
	a Web browser from different parts of the world and also from different devices
	(laptops, phones, tablets). An ability to access an application from different
	places and devices, without a need to install any additional software,
	has become the main feature of modern applications. Web applications reduce a
	complexity of accessing products and services and make them more attractive to an end
	user.
	
	Static HTML(HyperText Markup Language) Web pages, with a little amount of
	Javascript, which were constituting the big part of the Web are loosing their popularity.
	Modern Web	applications are very interactive and dynamic, they are becoming
	more powerful, and the difference between desktop and Web applications
	disappears. 
	
	Web technologies are developing so fast, that even such domain
	specific applications as Integrated Development Environments(IDE), trading
	systems or graphic editors can be accessed via a Web browser. As a result of
	the growth of Web applications, development and testing such complicated
	systems becomes more challenging.
	
	Especially in agile development when work is done in small iterations or
	cycles, changes in code require changes in testing. This gives a fast feedback and
    an opportunity to find and fix problems early, but also brings frustration
    for developers that they have to fix problems both in code and tests. 
	That might bring a false attitude that writing tests on early stages of the project,
	 when there is no clear picture of the final product,
	 increases the amount of work for developers. 
 
	 We are sure that writing tests reduces an overall work,
	 even if these tests have to be changed often. Usually a good rule is that
	 every patch should add or edit at least one test suite, this also ease the
	 reviewers job, because a test suite explains what is the reason of the patch.
	
	In the thesis we will describe tools and methodologies used for Web testing.
	Mainly we will focus on a Java-base framework for developing Web applications
	called Vaadin and a testing tool called Vaadin Testbench.
	
	 In the Fall of 2014 our team developed Vaadin Testbench
	 4.0.0 and released it in the beginning of December. The goal was to provide a
	 tool for a developer, that would help to write tests which simulate user
	 actions on the Web page. The main challenge was that code written with Vaadin
	 executes both on a client and a server side and a testing tool should handle
	 all the complexity in testing client-side and server-side communications. Another challenge was
	 to develop an universal and easy to use testing tool for Vaadin framework with a
	 clear Application Programming Interface(API).
	 
	 Our work is based on a previous version of Vaadin Testbench. In the thesis
	 we describe reasons for developing this testing tool and benefits of using it.
	 The thesis includes analysis and documentation of the performed work,
	 presenting main ideas and challenges of Web testing and how they were
	 solved during Testbench development.
	 
	 The result of the work was Testbench 4.0.0 released in December 2014.
	 Several user tests have shown, that a person with experience in Java and
	 Vaadin, but without any experience using Testbench, needs 15 minutes to setup
	 the environment and write a simple ``button-click'' test. We consider this
	 result as a success.
	  
	 Vaadin is an open-source project, but Testbench is a commercial
	 addon, nevertheless, there is a 30 days trial period, so anyone
	 can try Testbench and take a look at results of our work.
	  
	  \iffalse
		  I will also , because Testbench is focused on testing Web
		 applications written with Vaadin. I will also describe the working flow, what
		 tools and methodologies the team used and how the final product helps
		 Vaadin developers.
	  \fi 

	  The thesis is structured as following. Chapter \ref{ch:background} describes
	  Web applications and introduces Vaadin framework. Chapter \ref{ch:Webtesting}
	  describes and compares different techniques used in Web testing. Chapter
	  \ref{ch:selenium} represents Selenium - testing tool for Web applications.
	  Chapter \ref{ch:reasontestbenchdevelopment} shows the reasons for
	  developing Vaadin TestBench testing tool. Chapter \ref{ch:testbenchdevelop} presents the developing process of TestBench tools and methodologies used.
	  Chapter \ref{ch:testbenchvsselenium} compares TestBench and Selenium testing
	  tools. Chapter \ref{ch:testbenchuse} shows the example of writing tests and
	  the value for end user from using Testbench. 
	  Chapter \ref{ch:conclusion} summarize the whole thesis and list the results.