	
	 \chapter{Introduction}
	 \label{ch:intro} 		
	The 21st century has become an era of web applications. Software systems
	developed as a web based applcation allowes the end user to access data via web browser
	from different parts of the world and also from different devices (laptop,
	phones,tablets) has become one of the main features of modern
	applications. Static html web sites, with little amount of javascript, which
	were constituting the big part of the Web are loosing their popularity.
	
	 Modern web	applications are very interactive and dynamic, they are becoming
	more powerful, and the difference between desktop and web applications
	dissapears. Web technologies are developing so fast, that even such domain
	specific applications as IDE(Integrated Development Environment), trading
	sytems or graphic editors can be accessed via web browser. As a resut of the
	growth of web applications, development and maintanance of such complicated
	systems becomes more challenging. To reduce development and maintanance costs
	software frameworks are coming into existence.
	
	In this paper I will describe tools and methodologies used for web testing. Mainly I will focus on a Java-base
	framework for developing Web applications called Vaadin and a testing tool
	called Vaadin Testbench.
	
	 In the Fall of 2014 I was a part of the team which developed Vaadin Testbench
	 4.0.0 and released it in the beginging of December. Web testing tools is a
	 new topic and I will represent the main ideas and challenges
	 of web testing and how they were solved during Testbench development.
	 
	 The goal of this work is to provide a tool for a developer, that will help to
	 write tests which can simulate user actions on the web page. The main
	 challenge is that code written with Vaadin code executes both on the
	 client-side and server-side and testing tool should handle all the complexity
	 in testing client-side and server-side communications. Another challenge is to
	 develop an universal and easy to use testing tool for Vaadin framework with a
	 clear API.
	 
	  The result of the work was Testbench 4.0.0 released in December 2014.
	  Several user tests have shown, that a person with experience in Java and
	  Vaadin, but without any experience using Testbench, needs 15 minutes to setup
	  the envirenment and write a simple ``button-click'' test. We consider this
	  result as a success.
	  
	  Vaadin is an open-source project and Testbench is available with free for non-comercial use license. 
	  So every person from Vaadin community can try Testbench and take a look on results of our work and
	   decide does it suits his/her own needs.
	  
	  	 
	  I will also , because Testbench is focused on testing web
	 applications written with Vaadin. I will also describe the working flow, what
	 tools and methodologies the team used and how the final product helps
	 Vaadin developers.
	  
	  
	  This document is structured as follows. Chapter~\ref{ch:webtesting} describes
	  and compares different tehniques used in Web testing. Chapter~\ref{ch:vaadin} describes Vaadin framework.
	  Chapter \ref{ch:testbenchdevelop} presents the developing process of TestBench tools and methodologies used. Chapter
	  \ref{ch:testbenchuse} shows the example of writing tests and the value for
	  end user from using Testbench. Chapter~\ref{ch:concl} summarize the whole
	  document and list the results. There are two example appendices as well (Appendix A and B).
% ~\ref{app:A} and \ref{app:B}). Labels do not with \chapter*