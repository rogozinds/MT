	
	 \chapter{Introduction}
	 \label{ch:intro} 		
	The 21st century has become an era of Web applications. Software systems
	developed as a Web based application allows the end user to access data via
	Web browser from different parts of the world and also from different devices
	(laptops, phones,tablets). Ability to access an application from different places and
	devices, without need to install any additional software,
	has become the main feature of modern applications. Web applications reduces a
	complexity of accessing products and services and make them more attractive to an end
	user.
	
	Static HTML Web sites, with a little amount of Javascript, which
	were constituting the big part of the Web are loosing their popularity.
	Modern Web	applications are very interactive and dynamic, they are becoming
	more powerful, and the difference between desktop and Web applications
	disapears. 
	
	Web technologies are developing so fast, that even such domain
	specific applications as Integrated Development Environments(IDE), trading
	systems or graphic editors can be accessed via Web browser. As a result of the
	growth of Web applications, development and maintenance of such complicated
	systems becomes more challenging. To reduce development and maintenance costs
	software frameworks are coming into existence.
	
	In the thesis I will describe tools and methodologies used for Web testing.
	Mainly I will focus on a Java-base framework for developing Web applications called Vaadin and a testing tool
	called Vaadin Testbench.
	
	 In the Fall of 2014 I was a part of the team which developed Vaadin Testbench
	 4.0.0 and released it in the beginning of December. Web testing tools is a
	 new topic and I will represent the main ideas and challenges
	 of Web testing and how they were solved during Testbench development.
	 
	 The goal of this work is to provide a tool for a developer, that will help to
	 write tests which can simulate user actions on the Web page. The main
	 challenge is that code written with Vaadin executes both on the
	 client-side and server-side and testing tool should handle all the complexity
	 in testing client-side and server-side communications. Another challenge is to
	 develop an universal and easy to use testing tool for Vaadin framework with a
	 clear API.
	 
	  The result of the work was Testbench 4.0.0 released in December 2014.
	  Several user tests have shown, that a person with experience in Java and
	  Vaadin, but without any experience using Testbench, needs 15 minutes to setup
	  the environment and write a simple ``button-click'' test. We consider this
	  result as a success.
	  
	  Vaadin is an open-source project, but Testbench is a commercial
	  addon, nevertheless, there is a 30 days trial period, so anyone
	  can try Testbench and take a look on results of our work.
	  
	  \iffalse
		  I will also , because Testbench is focused on testing Web
		 applications written with Vaadin. I will also describe the working flow, what
		 tools and methodologies the team used and how the final product helps
		 Vaadin developers.
	  \fi

	  The thesis is structured as following. Chapter \ref{ch:Webtesting}
	  describes and compares different techniques used in Web testing.
	  Chapter \ref{ch:vaadin} describes Vaadin framework.
	  Chapter \ref{ch:testbenchdevelop} presents the developing process of
	  TestBench tools and methodologies used. Chapter \ref{ch:testbenchuse} shows the example of writing tests and the value for
	  end user from using Testbench. Chapter \ref{ch:conclusion} summarize the whole
	  thesis and list the results.