\chapter {Conclusion}
\label{ch:conclusion}
 The thesis describes Vaadin - a Java-based framework for Web applications and
testing tool called TestBench.
 According to analysis of an existing test tools and techniques used in for Web
applications we justify a demand of creating a testing tool specifically
for Vaadin applications. 

The thesis shows the advantage of using programmable tests against C\&R tests,
because of their better maintainability. Testing frameworks, such as Selenium,
allows to write test scripts which can be executed in a browser and simulate
user actions in the Web application. Selenium provides a wide browser support
and hides the differences between Web browser implementation behind the
WebDriver interface, which allows to navigate on the Web page and simulate user
actions.

Vaadin is a stateful framework and event on the client side may affect the state
of the application on the server side. Selenium is a testing tool on the client
side and it does not provide any support for Vaadin specific features, like
client-server communication. To develop tests for Vaadin applications using
Selenium developers need to understand Vaadin framework internal details and 
write extra code for synchronizing client and server sides of Vaadin components.
TestBench tool solves this problem and provides a convenient API for testing Vaadin applications.

Selenium operates on the DOM of the Web page, that is why writing tests for
Vaadin applications requires understanding of how the Vaadin component is
built to prepare a selector to find a Vaadin component or part of it. TestBench
provides an alternative selector variant with a greater abstraction for Vaadin components.
This makes it easier and faster to write tests and protects test scripts from potential
changes made to the client side implementation of Vaadin components.

TestBench provides extra features like screenshot comparison and
parallel test execution, and bring extra value for an end user in compare with
Selenium. Besides Vaadin Eclipse plugin makes starting using Vaadin Testbench
very easy. User case studies have shown, that a Java or Vaadin expert Vaadin, without any experience using TestBench, needs 15 minutes to setup the
 environment and run a simple test.

After six weeks of development our team TestBench 4.0.0 released in December
2014.  Vaadin TestBench is used both during Vaadin framework development and
during development Web applications based on Vaadin framework. Vaadin TestBench is a commercial product
 and distributed via CVAL3 license,  nevertheless every one can see try it during 30 days trial period.
 
 We are proud of the results of our work, because we got positive user feedback
 both in personal and online. For example, a user ``bill McCroskey'' left the
 following comment at Vaadin Web site: ``I have been very happy with the use of
 the Testbench and the relative ease of using the debugger to identify locators
  along with the Testbench utilities. Also use of Vaadin Id's to identify your components
  is essential for easy identification. We have been using Testbench for almost 
  a year now and we went from 0\% code coverage to over 70\%''\cite{vaadinBlog}.


 
