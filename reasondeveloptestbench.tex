\chapter{Reason for developing Testbench}
\label{ch:reasontestbenchdevelopment}
As shown in chapter~\ref{ch:selenium} Selenium is a powerful tool for Web
testing. However when using Selenium to test Vaadin application we will run into
several problems. Due to these reasons a test tool for Vaadin called TestBench was
created. TestBench is based on Selenium WebDriver, it solves  problems with client server communication,
 and brings more suitable API for working with Vaadin components

\section{Client server communication}
Selenium does not know about Vaadin specific features, like client-side	communication. 
Vaadin is a stateful framework, an event on a client side my affect a state of
the application on a server side. This brings additional complexity in testing,
because client-side and server-side states should be synchronized.

When event happens on the client side it will notify the server side.
If this event affects the server side state, the
server side will notify the client side about this change.
 Because of a network delay or long time code execution on the server side,
 there might be a delay between client side action and the change on a client. 
 In these circumstances the client-side should wait for server side code to execute,
  because it might affect the next client side instruction. To handle this
  situation we need to add a delay in a Selenium test, see line 20
   in listing~\ref{lst:seleniumVaadin} in appendix \ref{appendixA}. Rely on a
   delay in a test is a bad practise, because small delay might be not
   enough, adding a big delay on the contrary, will increase the test time
   execution.

\section{Extra code}
 Vaadin has a rich collection of UI components, but Selenium provides only
 basic methods like ``click'' or ``sendKeys'', which makes developers to right
 extra code for operating on Vaadin components. Same test for a text field
 written with Selenium \ref{lst:seleniumVaadin} and Vaadin
 TestBench\ref{lst:testbenchVaadin} takes 67 and 31 lines of code respectively.
   
