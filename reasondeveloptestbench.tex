\chapter{Reason for developing Testbench}
\label{ch:reasontestbenchdevelopment}
Selenium helps to automate Web applications for testing purposes. Large browser
support and open source make Selenium very popular for testing Javascript applications
and also a core technology for other automation testing tools. However when
using Selenium to test Vaadin application we will run into several challenges.

\section{Client server communication}
Vaadin is a stateful framework, an event on the client side may affect a state
of the application on the server side. Selenium is a testing tool on the client side, 
so it does not know about Vaadin specific features, like client-server communication. 

When an event happens on the client side, it will notify the server side through RPC.
If the RPC call caused a change in the shared state, it will send those changes to the client.
Because of a network delay or long time code execution on the server side, 
there might be a delay between the client side event and the change in the UI.
In these circumstances the client should wait for the server side code to execute,
because it might affect the next client side instruction.  
It is typical situation in Selenium that you need to add a delay between one instruction and the next.

Rely on a delay may cause a working test fail, because the delay was too small,
after facing such problems several times, developers start to add bigger delays,
which will increase the test time execution. 

\section{Extra code}
 Vaadin has a rich collection of UI components, but Selenium provides only
 basic methods like ``click'' or ``sendKeys'', which forces developers to write
 extra code for operating on Vaadin components. Same test for a text field
 written with Selenium  and Vaadin TestBench takes 67 and 31 lines of code
 respectively see examples in the Appendix A \ref{lst:seleniumVaadin} and
 \ref{lst:testbenchVaadin} .
   
 Due to these reasons a test tool for Vaadin called TestBench was created.
 TestBench is based on Selenium WebDriver, it solves  problems with client server communication,
 and brings more suitable API for working with Vaadin components.
