\chapter{Theoretical background}
\label{ch:background} 
	\section{Terms and Definitions}
		21st century has become an era of web applications. Software systems developed
		as a web based applcation allowes the end user to access data via web browser
		from different parts of the world and also from different devices (laptop,
		phones,tablets) has become one of the main features of modern
		applications. 
		
		
	  Static HTML Web sites are loosing their popularity, because users
	  expect from modern Web sites more than just pictures and text. Generally,
	  users willing to have a higly responsive application with different useful
	  features, working in the web. As a result Web applications are displacing Web
	  sites on the market. The difference of Web application from a Web site is the
	 “ability of a user to affect the state of the business logic on the server”[7]. In other words
	  the user/client makes a request to the server. The server perform some
	  actions (calculate, fetch data from database or external web-service) and
	  sends the response back to the client, which is rendered in the browser.
		
		
		Static html web sites, with little amount of javascript, which were
		constituting the big part of the Web are passing away. Modern web
		applications are very interactive and dynamic, they are becoming
		more powerful, and the difference between desktop and web applications
		dissapears. Web technologies are developing so fast, that even such domain
		specific applications as IDE(Integrated Development Dnvironment), trading
		sytems or graphic editors can be accessed via web browser.
		
		 The key concept that helps such complicated software to become web-based is a
		 multi-tier arhcitecture - the concept where the parts of the system are divided into
		separate tiers. This allows to develop presentation tier, which is
		responsible for user interface generation and lightweight validation, to
		be separate from complicated business-logic which runs on the server
		side. As a consequence the presentation tier code may be executed on different
		platforms, including web browsers. 
		
		As an resut of the growth of web applications the development and maintanance
		of  such complicated systems becomes more challenging.	All
		applications have a lot of common features and problems which	were already solved by developers beforehand. This is a good practise not to
		try to reenvent a whell, but take an already made solution. That is why many
		modern applications are based on one or several software frameworks. Indeed it is hard to imagine that some developers team will pick a programming
		language and start to develop everything from scratch, without using any
		framework or third-party library. Same implications are applyied for testing
		frameworks. The rapidly changing and highly competitive business environment,
		choosing a right toolset is one of the key factors of the sucess. 
		
		Nowdays some companies are still rely on manual testing or ignore this
		important part of software development at all. Such approach has several
		sorrowful consequences:
		
		1. The developers are affraid of changing already written
		code. Because they do not have a confidence that their changes will break something. They
			stop cleaning their production code because they fear the changes would
			do more harm than good. ``Their production code began to rot"
			\cite[p.123]{cleanCode} 
		2. The effort of finding errors and
			fixing them raises with the amount of code written. Because the developers can not localize the place where the error is
		actual happening.
		3. Developing new features become harder,if they are based on the part of the
		system which have errors.
		
		4. All in all this leads to increasing the cost of the whole system.
	 	
	 	 To test easily the huge amount code an automated web testing is come into
	 	 existence.
	 	 Web testing is a kind of software testing that accentuate on web which assists to slice down price,
	     lessen the exertion requisite to check web applications as well as web
	      sites, amplify software value, condense time-to-market and reusability of
	      test cases are also be done.
	   
		IEEE has defined software testing as the process of evaluating a software
		system to verify that it satisfies specified requirements [3 XU]. A set of
		requirments for the web application includes security, performance,
		presentation, etc. We will focus on several requirements for the web
		application which differ from desktop application. 
		
		One of the key requirements which makes testing web applications harder than
		testing desktop applicatiosn is support of different browsers and operating systems and also
		different devices. A lot of desktop applications are developed to support some
		particular operating system or different versions of the product are developed
		and maintained for different operating systems. Web applications on the
		contrary should support not only different operating systems, but also
		different browsers and devices. So, if developers team decides to support
		three operating systems (Windows, OSX, Android), three type of devices (phone,
		tablet, PC) and three browsers (Chrome, Firefox, Internet Explorer) the number
		of possible variations is already twenty seven. If you decide to support
		different version of browsers, which in some circumstances may vary a lot,
		the number of different configurations of tested machines will be close to
		one hudred. In this case manual testing is unexceptable, because it will lead
		to unwarranted expenses. 
		
		Another difference between web and desktop applications is
		navigation on the webpage and between pages, the unexpected state change via
		the browser Back button or direct URL entry in the browser. Also some
		resources or parts of the application can be not acccessable, due to
		connection problems or maintanance. Such unexpected behaviour may happen, and
		must be handled properly, not to crash the whole application.

		Web testing includes the different type of testing like:
			- functionality tests
			- compatibility tests
			- load tests
			- performance tests
			- integration tests
		All these types of tests are equivalent important and picking a tool which
		will help to write these tests is not an easy task. It is an advantage when
		the testing tool is using same principles and similiar programming language
		with other tools in the project. We think that using same programming language
		to write both tests and code is much easier for the developer. This idea is
		related to Test-Driven Development (TDD), when tests are written
		before production code.
		
		Test-Driven Development is a very popular methodology of software
			development. The main idea is to write tests first and then code. The main
			benefits of such approach are:
				1. The developer is sure that his code works as intendent, because all his
				code is tested.
				2. The errors are found at early stage of the development cycle, which
				reduces the cost of fixing problems.
			
			Three laws of TDD \cite[pp122]{Cleancode}[Book page 122]
				1. You may not write production code until you have written a failing unit
				test.
				2. You may not write more of a unit test than is sufficient to fail.
				3. You may not write more production code than is sufficient to pass the
				curently failing test.
			\iffalse	
					\subsection {Approaches in Web Testing}	
						\begin{textit}
							In this chapter I will explain
							what does testing actually meands. What types of testing exist: unit, integration, user-interface, regression,
						etc. What are the differences between these types of testing.
						
						Next I will prove why testing is so important in software development. Here I
						would like to mention some information about \textit{Quality control}. The
						idea is to show that testing increseases the speed of software development
						and also improves it's quality. So in terms of quality control testing will
						decrease the price and increase the value of the product for the end user.
						
						Also I want to mention other methodologies like \textit{Agile development},
						\textit{User experience design} and \textit{Test Driven Development}. And how
						testing can be used/integrated with these methodologies/processes.
						\end{textit}
			\fi