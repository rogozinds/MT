\chapter{Testbench vs Selenium}
\label{ch:testbenchvsselenium}
In this chapter we will summarize advantages of using TestBench against
Selenium.

\section{API built specifically for Vaadin components}
Selenium operates on the DOM of the Web page and provides only basic methods of
Web elements like ``click'' or ``sendKeys''. TestBench provides a rich API for
Vaadin components, which allows to operate on bigger parts of components, for
 example you can get a row or cell of the Table by index
 see~\ref{lst:testbench1}.
  	\lstset{language=Java}
  	\begin{lstlisting}[caption=Get Vaadin Table cell Value,label={lst:testbench1}]
TableElement table = getTableElement(); String value = table.getRow(0).getCell(1).getValue();
	\end{lstlisting}
	
\section{Client server communication}
As mentioned in section~\ref{sec:reasontestbenchdevelopment} TestBench handles 
client server syncronization.

\section{Screenshot comparesment}
Since version 4.0.0 TestBench has an API for comparing screenshots. This 
feature was introduced to help users to test UI of the application. We believe
that User Experience (UX) is very important in Web applications, and an
important part of UX is look and feel of the application.

In Web applications styling is done with CSS or SCSS. The downside of CSS 
is that changing CSS rule for one selector may affect a lot of elements on the
web page. For example change the width or margin of one element may ruin an
appearance of the whole Web page.

Manual UI testing is very difficult, because of the two main challenges:
\begin{enumerate}
  \item After some time the tester looses his concentration and does not see
  errors
  \item Small details in applications with rich UI is hard to notice for a
  human. In other words if you have several textboxes and buttons in different
  tabs or windows in the application it is hard to notice that some of them are
  not aligned.
\end{enumerate}

Both these two problems can be solved with automatic screenshot comparesment.
ImageComparesment class has an overloaded compare mehtod which takes either an
image or a path to the reference screenshot. The comparesment is done in 16x16
blocks comparing RGB values of the every pixel in this block. If there are
differences in images these parts are videleni with color, so it is easier for a
tester to find out what was the problem with the test. TestBench test can also
be configured to automatically take screenshots of the failing tests, it also
helps developers to narrow the scope of the failing test. 
tester


\section{Parallel testing}

\section{Finding UI element with mouse}
