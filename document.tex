%!TeX encoding = ISO-8859-1
\documentclass[12pt,a4paper,english%,twoside,openright
]{tutthesis}
% Ensure the correct Pdf size (not needed in all environments)
\special{papersize=210mm,297mm}
%
% Define your basic information
%
\author{Dmitrii Rogozin}
\title{Testing in web/vaadin applications (draft)}      % primary title (for front page)
\thesistype{Master of Science thesis} % or Bachelor of Science, Laboratory Report... 
\examiner{Kari Systä} % without title Prof., Dr., MSc or such

% Special trick to use internal macros outside the cls file
% (e.g. \@author). Trick is reversed with \makeatother a bit later.
\makeatletter

% Define the pdf document properties.  Fill in your own keywords.
\hypersetup{   
  pdftitle={\@title},
  pdfauthor={\@author},
  pdfkeywords={Web-testing Automation testing Vaadin Selenium TestBench}
}
\usepackage[english]{babel}
%
% You can include special packages or define new commands here at the
% beginning. Options are given in brackets and package name is in
% braces:  \usepackage[opt]{pkg_name}
\DeclareGraphicsRule{*}{mps}{*}{}
\usepackage[export]{adjustbox}
\graphicspath{ {images/} }
\lstset{
  language=Java, % the language of the code
   basicstyle=\tiny                
}


% Preparatory content ends here



\pagenumbering{Roman}
\pagestyle{headings}
\begin{document}

% Create the title page.
% First the logo. Check its language.
\thispagestyle{empty}
%\vspace*{-.5cm}\noindent
\vspace*{-1cm}\noindent
\includegraphics[width=8cm]{tty_tut_logo}   % Bilingual logo

% Then lay out the author, title and type to the center of page.
\vspace{6.8cm}
\maketitle
\vspace{7cm} % modify if thesis title needs many lines

% Last some additional info to the bottom-right corner
\begin{flushright}  
  \begin{minipage}[c]{6.8cm}
    \begin{spacing}{1.0}
      %\textsf{Tarkastaja: Prof. \@examiner}\\
      %\textsf{Tarkastaja ja aihe hyv�ksytty}\\ 
      %\textsf{xxxxxxx tiedekuntaneuvoston}\\
      %\textsf{kokouksessa dd.mm.yyyy}\\
      \textsf{Examiner: Prof. \@examiner}\\
      \textsf{Examiner and topic approved by the}\\ 
      \textsf{Faculty Council of the Faculty of}\\
      \textsf{xxxx}\\
      \textsf{on 1st November 2015}\\
    \end{spacing}
  \end{minipage}
\end{flushright}

% Leave the backside of title page empty in twoside mode
\if@twoside
\clearpage
\fi
%
% Use Roman numbering i,ii,iii... for the first pages (abstract, TOC,
% termlist etc)
\pagenumbering{roman} 
\setcounter{page}{0} % Start numbering from zero because command 'chapter*' does page break


% Some fields in abstract are automated, namely those with \@ (author,
% title in the main language, thesis type, examiner).
\chapter*{Abstract}
\begin{spacing}{1.0}
         {\bf \textsf{\MakeUppercase{\@author}}}: \@title\\   % use \@titleB when thesis is in Finnish
         \textsf{Tampere University of Technology}\\
         \textsf{\@thesistype, xx pages, x Appendix pages} \\
         \textsf{xxxxxx 201x}\\
         \textsf{Master's Degree Programme in xxx Technology}\\
         \textsf{Major: }\\
         \textsf{Examiner: Prof. \@examiner}\\ % 
         \textsf{Keywords: }\\
\end{spacing}

The abstract is a concise 1-page description of the work: what was the
problem, what was done, and what are the results. Do not include
charts or tables in the abstract.(100-150 words)



\makeatother % Make the @ a special symbol again, as \@author and \@title are not neded after this
\chapter*{Preface}
This document template conforms to Guide to Writing a Thesis at
Tampere University of Technology (2014) and is based on the previous
template. The main purpose is to show how the theses are formatted
using LaTeX (or \LaTeX ~ to be extra fancy) .

The thesis text is written into file \texttt{d\_tyo.tex}, whereas
\texttt{tutthesis.cls} contains the formatting instructions. Both
files include lots of comments (start with \%) that should help in
using LaTeX. TUT specific formatting is done by additional settings on
top of the original \texttt{report.cls} class file. This example needs
few additional files: TUT logo, example figure, example code, as well
as example bibliography and its formatting (\texttt{.bst}) An example
makefile is provided for those preferring command line. You are
encouraged to comment your work and to keep the length of lines
moderate, e.g. <80 characters. In Emacs, you can use \texttt{Alt-Q} to
break long lines in a paragraph and \texttt{Tab} to indent commands
(e.g. inside figure and table environments). Moreover, tex files are
well suited for versioning systems, such as Subversion or Git.  
% \url{http://www.ctan.org/tex-archive/info/lshort/english/lshort.pdf}


Acknowledgements to those who contributed to the thesis are generally
presented in the preface. It is not appropriate to criticize anyone in
the preface, even though the preface will not affect your grade. The
preface must fit on one page. Add the date, after which you have not
made any revisions to the text, at the end of the preface.

~ 
% Tilde ~ makes an non-breakable spce in LaTeX. Here it is used to get
% two consecutive paragraph breaks

Tampere, 1.9.2015

~
Dmitrii Rogozin
%
% Add the table of contents, optionally also the lists of figures,
% tables and codes.
%

\renewcommand\contentsname{Table of Contents} % Set English name (otherwise bilingual babel might break this), 2014-09-01
%\renewcommand\contentsname{Sis�llys}         % Set Finnish name
\setcounter{tocdepth}{3}                      % How many header level are included
\tableofcontents                              % Create TOC

\renewcommand\listfigurename{List of Figures}  % Set English name (otherwise bilingual babel might break this)
%\renewcommand\listfigurename{Kuvaluettelo}    % Set Finnish name
\listoffigures                                 % Optional: create the list of figures
\markboth{}{}                                  % no headers

\renewcommand\listtablename{List of Tables}    % Set English name (otherwise bilingual babel might break this)
%\renewcommand\listtablename{Taulukkoluettelo} % Set Finnish name
\listoftables                                  % Optional: create the list of tables
\markboth{}{}                                  % no headers


%\renewcommand\lstlistlistingname{List of Programs}      % Set English name (otherwise bilingual babel might break this)
%%\renewcommand\lstlistlistingname{Ohjelmaluettelo} % SetFinnish name, remove this if using English
%\lstlistoflistings                                % Optional: create the list of program codes
%\markboth{}{}                                     % no headers


%
% Term and symbol exaplanations use a special list type
%

\chapter*{List of abbreviations and symbols}
%\chapter*{Lyhenteet ja merkinn�t}
\markboth{}{}                                % no headers

% You don't have to align these with whitespaces, but it makes the
% .tex file more readable
\begin{termlist}
\item [API] Application Programming Interface
\item [BDD] Behaviour-Driven Development
\item [DOM] Document Object Model
\item [GWT]Google Web Toolkt
\item [JVM] Java Virtual Machine
\item [HTML] HyperText Markup Language
\item [XHTML] EXtensible HyperText Markup Language
\item [C\&R]Capture and Replay
\item [IDE] Integrated Development Environment
\item [JRE] Java Runtime Envirenment
\item [POM] Project Object Model
\item [UI] User Interface
\item [URL] Uniform Resource Locator

\end{termlist} 


% Leave the backside of abbreviation list empty in twoside mode
\cleardoublepage

% The actual text begins here and page numbering changes to 1,2...
\newpage             % needed for page numbering
\pagenumbering{arabic}
\setcounter{page}{1} % Start numbering from zero because command
                     % 'chapter*' does page break
\renewcommand{\chaptername}{} % This disables the prefix 'Chapter' or
                              % 'Luku' in page headers (in 'twoside'
                              % mode)

	
	 \chapter{Introduction}
	 \label{ch:intro} 		
	 Web applications provide critical services to our society,
	 ranging from the financial and commercial sector,to the public
	 administration and health care.The wide spread use of web
	 applications as the natural interface between a service and its
	 users puts a serious demand on the quality levels that web
	 application developers are expected to deliver. At the same time,
	 web applications tend to evolve quickly, especially for what
	 concerns the presentation and interaction layer. The release
	 cycle of web applications is very short, which makes it difficult
	 to accommodate quality assurance (e.g.,testing) activities in 
	 the development process when a new release is delivered. For
	 these reasons, the possibility to increase the effectiveness and
	 efficiency of web testing has become a major need and several 
	 methodologies, tools and techniques have been developed over time.
	 
	 In these paper I will present Vaadin framework and testing tool for Vaadin,
	 called Vaadin Testbench.
	
	``Vaadin Framework is a Java web application development framework that is
	designed to make creation and maintenance of high quality web-based user interfaces easy.
	 Vaadin supports two different programming models: server-side and client-side. 
	 `` \cite[pr1.1]{bookVaaidn}
	 Vaadin TestBench is a tool for automated user interface testing of web
	 applications on all platforms and browsers. \cite{vaadinTestbenchSite}
	 
	 In the Fall 2014 I was a part of the team which developed Vaadin Testbench
	 4.0.0 and released it in the beginging of December. Web testing tools is a
	 new topic and in the this work I will represent the main ideas and challenges
	 of web testing and how they were solved during Testbench development. I will
	 also describe Vaadin framework, because Testbench is focused on testing web
	 applications written with Vaadin. I will also describe the working flow, what
	 tools and methodologies were used and how the final product might help Vaadin
	 developers.
	 
	 The goal of this work is to provide a tool for a developer, that will help to
	 write tests which can simulate user actions on the web page. The main
	 challenge is that code written with Vaadin might execute both on the
	 client-side and server-side, so that events happen on the client-side will be
	 properly sent to the server-side. Another challenge is to develop an universal
	 easy to use testing tool for Vaadin framework with a clear API.
	 
	  The result of the work was Testbench 4.0.0 released in December 2014.
	  Several user tests have shown, that a person with experience in Java and
	  Vaadin, but without any experience using Testbench, needs 15 minutes to setup
	  the envirenment and write a simple ``button-click'' test. We consider this
	  result as a success. Vaadin is an open-source project and Testbench is
	  available with free for non-comercial use license. So every person from Vaadin community can try
	  Testbench and take a look on results of our work and decide does it suits
	  his/her own needs.
	  
	  This document is structured as follows. Chapter~\ref{ch:style}
	  discusses briefly the basics of writing and presentation style
	  regarding the text, figures, tables and mathematical
	  notations. Chapters~\ref{sec:ref_styles} and \ref{ch:concl} summarize
	   the referencing basics and the whole document. There are two example
	appendices as well (Appendix A and B).
% ~\ref{app:A} and \ref{app:B}). Labels do not with \chapter*
% \label{...} allows cross-referencing, e.g. 'as explained in
% Chapter~\ref{ch:intro}' Note that you may have to run the command
% 'latex' or 'pdflatex' twice to get cross-references correctly.  You
% can add labels e.g. to chapters, sections, figures, tables, and
% equations.

% You can write everything into single tex file. Alternatively, you
% can write each chapter into separate file and then include them her

\chapter{Theoretical background}
\label{ch:background} 
	\section{Terms and Definitions}
		21st century has become an era of web applications. Software systems developed
		as a web based applcation allowes the end user to access data via web browser
		from different parts of the world and also from different devices (laptop,
		phones,tablets) has become one of the main features of modern
		applications. 
		
		
	  Static HTML Web sites are loosing their popularity, because users
	  expect from modern Web sites more than just pictures and text. Generally,
	  users willing to have a higly responsive application with different useful
	  features, working in the web. As a result Web applications are displacing Web
	  sites on the market. The difference of Web application from a Web site is the
	 “ability of a user to affect the state of the business logic on the server”[7]. In other words
	  the user/client makes a request to the server. The server perform some
	  actions (calculate, fetch data from database or external web-service) and
	  sends the response back to the client, which is rendered in the browser.
		
		
		Static html web sites, with little amount of javascript, which were
		constituting the big part of the Web are passing away. Modern web
		applications are very interactive and dynamic, they are becoming
		more powerful, and the difference between desktop and web applications
		dissapears. Web technologies are developing so fast, that even such domain
		specific applications as IDE(Integrated Development Dnvironment), trading
		sytems or graphic editors can be accessed via web browser.
		
		 The key concept that helps such complicated software to become web-based is a
		 multi-tier arhcitecture - the concept where the parts of the system are divided into
		separate tiers. This allows to develop presentation tier, which is
		responsible for user interface generation and lightweight validation, to
		be separate from complicated business-logic which runs on the server
		side. As a consequence the presentation tier code may be executed on different
		platforms, including web browsers. 
		
		As an resut of the growth of web applications the development and maintanance
		of  such complicated systems becomes more challenging.	All
		applications have a lot of common features and problems which	were already solved by developers beforehand. This is a good practise not to
		try to reenvent a whell, but take an already made solution. That is why many
		modern applications are based on one or several software frameworks. Indeed it is hard to imagine that some developers team will pick a programming
		language and start to develop everything from scratch, without using any
		framework or third-party library. Same implications are applyied for testing
		frameworks. The rapidly changing and highly competitive business environment,
		choosing a right toolset is one of the key factors of the sucess. 
		
		Nowdays some companies are still rely on manual testing or ignore this
		important part of software development at all. Such approach has several
		sorrowful consequences:
		
		1. The developers are affraid of changing already written
		code. Because they do not have a confidence that their changes will break something. They
			stop cleaning their production code because they fear the changes would
			do more harm than good. ``Their production code began to rot"
			\cite[p.123]{cleanCode} 
		2. The effort of finding errors and
			fixing them raises with the amount of code written. Because the developers can not localize the place where the error is
		actual happening.
		3. Developing new features become harder,if they are based on the part of the
		system which have errors.
		
		4. All in all this leads to increasing the cost of the whole system.
	 	
	 	 To test easily the huge amount code an automated web testing is come into
	 	 existence.
	 	 Web testing is a kind of software testing that accentuate on web which assists to slice down price,
	     lessen the exertion requisite to check web applications as well as web
	      sites, amplify software value, condense time-to-market and reusability of
	      test cases are also be done.
	   
		IEEE has defined software testing as the process of evaluating a software
		system to verify that it satisfies specified requirements [3 XU]. A set of
		requirments for the web application includes security, performance,
		presentation, etc. We will focus on several requirements for the web
		application which differ from desktop application. 
		
		One of the key requirements which makes testing web applications harder than
		testing desktop applicatiosn is support of different browsers and operating systems and also
		different devices. A lot of desktop applications are developed to support some
		particular operating system or different versions of the product are developed
		and maintained for different operating systems. Web applications on the
		contrary should support not only different operating systems, but also
		different browsers and devices. So, if developers team decides to support
		three operating systems (Windows, OSX, Android), three type of devices (phone,
		tablet, PC) and three browsers (Chrome, Firefox, Internet Explorer) the number
		of possible variations is already twenty seven. If you decide to support
		different version of browsers, which in some circumstances may vary a lot,
		the number of different configurations of tested machines will be close to
		one hudred. In this case manual testing is unexceptable, because it will lead
		to unwarranted expenses. 
		
		Another difference between web and desktop applications is
		navigation on the webpage and between pages, the unexpected state change via
		the browser Back button or direct URL entry in the browser. Also some
		resources or parts of the application can be not acccessable, due to
		connection problems or maintanance. Such unexpected behaviour may happen, and
		must be handled properly, not to crash the whole application.

		Web testing includes the different type of testing like:
			- functionality tests
			- compatibility tests
			- load tests
			- performance tests
			- integration tests
		All these types of tests are equivalent important and picking a tool which
		will help to write these tests is not an easy task. It is an advantage when
		the testing tool is using same principles and similiar programming language
		with other tools in the project. We think that using same programming language
		to write both tests and code is much easier for the developer. This idea is
		related to Test-Driven Development (TDD), when tests are written
		before production code.
		
		Test-Driven Development is a very popular methodology of software
			development. The main idea is to write tests first and then code. The main
			benefits of such approach are:
				1. The developer is sure that his code works as intendent, because all his
				code is tested.
				2. The errors are found at early stage of the development cycle, which
				reduces the cost of fixing problems.
			
			Three laws of TDD \cite[pp122]{Cleancode}[Book page 122]
				1. You may not write production code until you have written a failing unit
				test.
				2. You may not write more of a unit test than is sufficient to fail.
				3. You may not write more production code than is sufficient to pass the
				curently failing test.
			\iffalse	
					\subsection {Approaches in Web Testing}	
						\begin{textit}
							In this chapter I will explain
							what does testing actually meands. What types of testing exist: unit, integration, user-interface, regression,
						etc. What are the differences between these types of testing.
						
						Next I will prove why testing is so important in software development. Here I
						would like to mention some information about \textit{Quality control}. The
						idea is to show that testing increseases the speed of software development
						and also improves it's quality. So in terms of quality control testing will
						decrease the price and increase the value of the product for the end user.
						
						Also I want to mention other methodologies like \textit{Agile development},
						\textit{User experience design} and \textit{Test Driven Development}. And how
						testing can be used/integrated with these methodologies/processes.
						\end{textit}
			\fi 
	\chapter{Web testing}
	\label{ch:Webtesting}

		There are several approaches for Web testing, the choice among them depends on
		different factors such as a life cycle of the project, technologies used, the
		budget and a professional level of developers. Two main ideas are Capture
		and Replay tests (C\&R) and programmable tests.
		
\section{Capture and Replay}
\label{sec:captureReplay}
C\&R tools have been developed for testing the applications against graphical user interfaces. 
The software tester works with the Web application modeling user behaviour,
the C\&R tool records the whole session and generates a
script, which can be executed later,	repeating same actions without humans participation.
The main idea of C\&R
testing tools is to record a sequence of user actions, that can be
automatically replayed later and save them in a human readable format. A human
readable format is an important feature, because script editing might
be useful to adjust failed scripts accordingly	to the changes of the Web page.
Though if the Web page was changed	significantly, editing test script might be more expensive than recording a
new test\cite{CaptureReplay7}. 
			
C\&R tools are usually Web browser plugins or Java applets, because they
need to control a Web browser to navigate on a Web page. User actions are
usually saved as a sequence of steps in HTML or XML, which is later used to
generate a Javascript code to reproduce user actions. An example of C\&R testing
tool is Selenium IDE described in chapter\ref{ch:selenium}.

The main advantage of C\&R, is that the tester does not
require to have experience in coding. Building test cases with such tools
is a simple task.
The biggest downside of C\&R tests is that editing generated
scripts is harder than editing scripts written by a software
developer\cite{CaptureReplay7}. The test cases are strongly coupled with Web pages
 and contain hard-coded values. These factors lead to the
problem, that very often the tester have to record a new test, instead of
changing the existing one.

 When using C\&R tools the tester can
not use loose-coupling, decomposition or other design techniques to make
easy readable and maintainable tests. It is also hard to use parts of already
made test cases when creating new ones. Programmable tests can help to solve these problems.

\section{Programmable tests} 
\label{sec:programTests}
Programmable tests are created by a tester manually. This method requires the
person to have programming skills and takes more time, but programmable tests
are more flexible and allow the developer to use bigger set of tools. The
developer may use conditional statements to change execution of the test,
loops to repeat same actions, exception handling, data structures like
arrays, sets, trees, graphs, logging and etc. Programmable tests are more
flexible and powerful than C\&R tests and provide an opportunity to create
parameterized tests - tests which can be executed multiple times with different arguments. 

Thought writing programmable tests is harder than C\&R and requires more
 experience and skills, programmable tests provide more flexibility and scalability. The empirical study
       of developing tests for four different frameworks shows that the development of 
       programmable tests is more time
      consuming (between 32\% and 112\%), but test maintenance requires less
      time (with a saving of 16\% and 51\%). As a result ``In general, programmable test cases are more
      expensive to write but easier to evolve than C\&R ones, with an advantage
      after two releases (in the median case)''.\cite{CaptureReplay7}

 \section{Capture and Replay vs Programmable tests}
 To show that programmable tests are more powerful than C\&R we provide an
artificial example of testing Vaadin framework.	
Vaadin has a set of UI element classes and we want to	test that changing 
the value of the elements triggers a value change listener event.
First, we create a Web page where we add elements we
want to test such as a text field, a combo box and a radio button. Second, we
add a value change listener to all this elements that will change the value of a test field.
When the radio button state changes, event listener method
will change the value of the test field to ``radio button event triggered''. In the test we
iterate through all the elements, change their values and check the value
of the test field see example of test UI \ref{lst:vaadintest} and test case
classes \ref{lst:vaadintest2} in the Appendix A.

The biggest advantage of programmable tests against C\&R is scalability and
better maintainability. When using C\&R a tester should record same actions for
each new element, while with  programmable tests, testing new elements requires
small changes to an existing test case. 

\section {Challenges}
	\label {sec:challenges}
	\subsection{Look and feel testing}
	    Both C\&R and programmable tests are focused on a DOM of the Web page.
	    One disadvantage is they do not provide tools to test appearance of the Web
	    page like colors, margins, fonts, etc. The client side page may have bugs in CSS
	    or HTML, for example HTML elements with a ``display:none'' CSS rule would
	    not be shown for a user in the Web browser.
	    Though all user actions could be still emulated by a testing tool. 
	    This problem can be solved by adding screenshot
      comparison see \ref{sec:screencompare}, when a screenshot of a tested Web
      page is compared with a reference screenshot. 
	\subsection{Complex DOM structure}
    Real-life example may have dozens/hundreds of HTML
		elements on the Web page see figure
		\ref{fig:gmailexample}.
		Web pages with big and branched DOM bring several challenges:
		\begin{enumerate}
		  \item Searching for required element may be very resource-consuming,
		  affecting time of the test execution. 
		  \item Changes in DOM may require changes in tests, which increase
		  application maintenance costs.
		\end{enumerate}
		
		\begin{figure}
		\centering
		  \includegraphics[width=0.75\textwidth]{gmail_example}
		  \caption{Gmail page structure}
		  \label{fig:gmailexample}
		\end{figure}
		
		Testing frameworks allow several strategies for locating Web page elements:
		\begin{enumerate}
		  \item By id - locates the Web page elements using their id values.
		  \item By name - locates the Web page elements using their name.
		  \item By tag - locates the Web page elements using their tag.
		  \item By class - locates the Web page elements using their class attribute.
		  \item By XPath - combines previous strategies and builds a search
		  path to an element in the DOM.
		\end{enumerate}
		
		The choice between these strategies is a trade-off between efficiency of
		the test and its complexity for developer. An industrial case study shows that 
		XPath methods for locating elements on a Web page take 
		three times longer than same tests with searching
    elements by id\cite{selenium4}. 
	   Id-based tests for static HTML Web pages with small amount of elements
	   is an optimal solution. On the contrary, in case of dynamically
	   generated HTML with many elements as in example \ref{fig:gmailexample}
	   searching elements by id is inappropriate.
		
		The biggest downside of searching by id strategy, is that every HTML element
		should have a unique ID. If the Web pages has a dynamically generated content,
		for example a table, where amount of rows depends on data, the
		developer has to add some logic to generate ID for each row and also verify
		that new ids do not conflict with already created ones. 
		
		In some circumstances the developer needs to get a set of elements by some
		criteria, for example get all elements with a specific tag or class selector
		and process them in a loop. 
		
		As we can see there is no ideal solution for searching elements on a Web page,
		which can be used in all cases. The developer should decide which
		searching algorithm to use based on requirements, but the testing framework
		should provide the developer different tools to choose from. In chapter
		\ref{ch:selenium} we will present Selenium - a software testing framework
		for Web applications, which allows to create C\&R \ref{sec:captureReplay} and 
    programmable tests \ref{sec:programTests}. Selenium  supports searching
    elements by id,tag, class, XPath, etc and provides an opportunity simulate
    user actions on a Web page.
		

 
	\chapter {Selenium}
	\label{ch:selenium}
      Selenium is a set of different software tools each with a different approach
       to supporting test automation. The entire suite of tools allows many
       options for locating UI elements and comparing expected test results
       against actual application behavior \cite{seleniumSite}.
       Selenium provides implementation both of C\&R \ref{sec:captureReplay}
       \ref{sec:programTests} and programmable tests models.  
       
       \textbf{Selenium IDE} - is a development environment with graphical
       interface for building test scripts.
		Selenium IDE has a recording feature,  which records user actions as they are
		performed and then exports them as a reusable script in one of many programming languages that can be later executed.
       
       \textbf{Selenium  WebDriver} makes direct calls to the browser using browser's native support for automation. WebDriver represents a web browser, hiding specific browser details, behind the interface see
       figure~\ref{fig:webdriver}. Having a unified interface provides
       multi-browser support and allows same tests to be executed in different
       envirements.
       
	    \begin{figure}
		\label{fig:webdriver}
		\includegraphics[width=0.75\textwidth, center]{webdriver_structure}
		\caption{WebDriver structure}
		\end{figure}
		
    	\textbf{Selenium Remote Control (RC)} is an old version of Selenium
    	WebDriver, currently supported only in maintanance mode.
       
       \textbf{Selenium Grid} allows to execute tests on different machines.
       Selenium Grid is useful for projects with large amount of tests or test
       suites that must be run in multiple environments.
       
		Grid allows to add several physical machines to a test cluster. Grid uses the
		term ``hub'' for a central point where all tests are loaded. Hub is responsible
		for distributing the tests across nodes. ``Node'' is  a remote machine with
		specific configuration which is attached to the hub see
		figure~\ref{fig:selnium_grid}. Nodes are totally separated from each other and
		may have different operating systems and browsers. Hub ``decides''
		for each test suite on which node it should be executed based on test's configuration.
		By default every node starts eleven browsers : five Firefox, five Chrome and
		one Internet Explorer.
		
		\begin{figure}
		\label{fig:selnium_grid}
		\includegraphics[width=0.75\textwidth, center]{selenium_grid}
		\caption{Selenium Grid structure}
		\end{figure}
		
		Selenium Grid is published as a separate jar file, so to setup a test hub you
		only need to have JRE(Java Runtime Envirenment) installed. To start hub run
		selenium-server with hub parameter see \ref{lst:starthub} and to start a new
		node specify a webdriver parameter and the URL of the hub running \ref{lst:startnode}.
		
		\begin{lstlisting} [ label={lst:starthub},language=bash, caption=Start hub ]
		 java -jar	selenium-server-standalone-2.30.0.jar -role hub
		\end{lstlisting}

		\begin{lstlisting} [caption=Start node, label={lst:startnode},language=bash]
		java -jar selenium-server-standalone-2.30.0.jar -role webdriver
		-hub http://http://192.168.1.1:4444/grid/register -port 5566
		\end{lstlisting}
		

	To create a test suite for the Grid configuration created above see an
	example \ref{lst:starthub2}. Here is a explanation of basic steps :
	\begin{itemize}
	  \item Specify a url of tested web page line 6.
	  \item Specify a url of a hub line 7.
	  \item Specify settings of the test, for example a browser line 8.
	  \item Create an instance of remote webDriver line 9.
	  \item Open a tested web page line 14.
	  \item Compare actual value on a webpage with expected value line 15.
	  \item Close web page line 20.
	\end{itemize}
	
	\lstset{style=a1listing}
	\begin{lstlisting} [caption=Selenium test example,label={lst:starthub2},language=java]
public 	class TestExample throws MalformedURLException {
 	WebDriver driver;
 	 String UIUrl,nodeURL;
 	 @BeforeTest public void setUp() {
 	 	UIUrl="http://app.example.com/hellopage"; 
		hubURL="http://192.168.1.2:5566/wd/hub";
		DesiredCapabilities capability=DesiredCapabilities.firefox();
		driver=new RemoteWebDriver(new URL(hubURL),capability);
	}
		
	@Test
	public void test1() {
		driver.get(UIUrl);
		Assert.assertEquals("Welcome",driver.getTitle());
	}
	
	@AfterTest
	public void afterTest() {
		driver.quit();
	}
}
	\end{lstlisting}

	As it shown above setting up the basic configuration of Selenium Grid is fairly
	easy. After we have opened the tested web page with driver.get(UIUrl), you can
	use ``Find Element" or ``Find Elements" methods with ``By" query object for
	locating elements on the page. For example to find an element with a
	class ``profile'' you need to use By.className query object see example
	~\ref{lst:classSearch}. Selenium  supports searching elements by id,tag, class,
	Xpath and all implications from section~\ref{sec:challenges} applies to it. 
	There is no ideal strategy for searching the required element it is up
	to developer which one to choose based on requirements.
	
	\lstset{style=a1listing}
	\begin{lstlisting} [caption=Search element by class,label={lst:classSearch}]
	WebElement avatarElement = driver.findElement(By.className("profile"));
	\end{lstlisting}
\chapter{Reason for developing Testbench}
\label{ch:reasontestbenchdevelopment}
As shown in chapter~\ref{ch:selenium} Selenium is a powerful tool for web
testing.
However if we use Selenium to test Vaadin application we will run into several problems.

Selenium does not know about Vaadin specific features, like client-side	communication. 
Vaadin is a stateful framework - an event on a client side my affect a state of
the applicaton on a server side. This brings additional complexity in testing,
because client-side and server-side states should be synchronized.

When event happens on the client side it will notify the server side. If this event affects the server side state, the
server side will notify the client side about this change.
 Because of a network delay or long time code execution on the server side,
 there might be a delay between client side action and the change on a client. 
 In these circumstances the client-side should wait for server side code to execute,
  because it might affect the next client side instruction. To handle this
  situation we need to add a delay in a Selenium test, see line 20
   in listing~\ref{lst:seleniumVaadin}. Rely on a delay in a test is a bad
   practise, because small delay might be not enough to all client server
   communications to proceed. Adding big delay on the contrary, will increase
   the test time execution. 
   
 Because Selenium framework is not related to Vaadin it requires additional
   code to take into account client-server communication. For example setting
   value of the text field requires three lines of code instead of one see lines
   25-28 in listing~\ref{lst:seleniumVaadin}.
   
   Due to these implications a test tool for Vaadin called TestBench was
   created. TestBench is based on Selenium WebDriver, but it solves
   problems with client server communication, and brings more suitable
   API for working with Vaadin components. For example same test written with 
   the help of TestBench takes less code and does not have delays see example~\ref{lst:testbenchVaadin}.
  	
  	
  	\lstset{style=a1listing}
  	\begin{lstlisting} [caption=Selenium test for Vaadin application,label={lst:seleniumVaadin}]
public class AppTest extends TestCase{
	WebDriver driver;
    String UIUrl, nodeURL;	
	@Override @Before
    public void setUp() {
        UIUrl = "http://demo.vaadin.com/dashboard/";
        driver = new FirefoxDriver();

    }

    @Test
    public void testWithoutTestbench() {
        driver.get(UIUrl);
        driver.manage().timeouts().implicitlyWait(5, TimeUnit.SECONDS);
        List<WebElement> elements = driver.findElements(By
                .className("v-button"));
        if (elements.isEmpty()) {
            throw new RuntimeException("No buttons found");
        }
        elements.get(0).click();
        driver.findElement(By.id("dashboard-edit")).click();
        WebElement searchField = driver.findElements(By.className("v-textfield")).get(0);
        searchField.clear();
        searchField.sendKeys("New Dashboard");
        searchField.sendKeys(Keys.TAB);
        WebElement searchButton = findButtonByCaption("Save");
        searchButton.click();
        driver.manage().timeouts().implicitlyWait(5, TimeUnit.SECONDS);
        String title = driver.findElement(By.id("dashboard-title")).getText();
        Assert.assertEquals("New Dashboard", title);
    }

    public WebElement findButtonByCaption(String caption) {
        List<WebElement> buttons = driver
                .findElements(By.className("v-button"));
        for (WebElement button : buttons) {
            if (button.getText().equals(caption)) {
                return button;
            }

        }
        return null;
    }

    public WebElement findButtonByCaption(WebElement parent, String caption) {
        List<WebElement> buttons = parent
                .findElements(By.className("v-button"));
        for (WebElement button : buttons) {
            if (button.getText().equals(caption)) {
                return button;
            }

        }
        return null;
    }

    @After
    public void afterTest() {
        driver.quit();
    }
}
\end{lstlisting}

	
	\ref{lst:testbenchVaadin}.
	\lstset{style=a1listing}
  	\begin{lstlisting} [caption=TestBench test,label={lst:testbenchVaadin}]
public class TestBenchTest extends TestBenchTestCase {

    WebDriver driver;
    String UIUrl, nodeURL;

    @Before
    public void setUp() throws Exception {
        UIUrl = "http://demo.vaadin.com/dashboard/";
        setDriver(new FirefoxDriver());
    }

    @Test
    public void test1() {
        getDriver().get(UIUrl);
        $(ButtonElement.class).first().click();
        $(ButtonElement.class).id("dashboard-edit").click();
        TextFieldElement searchField = $(TextFieldElement.class).first();
        searchField.setValue("New dashboard");
        $(ButtonElement.class).caption("Save").first().click();

        String title = $(LabelElement.class).id("dashboard-title").getText();
        Assert.assertEquals("New dashboard", title);
    }

    @After
    public void afterTest() {
        // driver.quit();
    }
}  	
\end{lstlisting}
\chapter{Development of Testbench}
\label{ch:testbenchdevelop}
  
  The estimated time for developing Testbench4 was from two months to six weeks.
  Our team had four members:
  \begin{itemize}
    \item Anthony Guerreiro - developer.
    \item Dmitrii Rogozin - developer.
    \item Jonatan Kronqvist - tech lead.
    \item Mika Mutajarvi - developer.
    
  \end{itemize}
  
   This is a short period of time and to manage delivering a good-quality product you
    have to minimize overhead costs. We believe that a team should choose tools
    and methodology which suits its purposes. 
    Our team decided to use Scrum and Test Driven Development(TDD) for managing
    product development and try to be agile and flexible. 
    We decided to have two week sprints.

  \section{Scrum}
    Scrum is a management and control process that cuts through
    complexity to focus on building software that meets business needs. Management and teams are able to get their
    hands around the requirements and technologies, never let go, and deliver working software,
    incrementally and empirically. The Scrum Team consists of a Product Owner,
    the Development Team, and a Scrum Master.
    
  \textbf{Product Owner}(PO) - decides what features should the product have to
  maximally increase the satisfaction of the end user of the product and puts this features to the backlog.
  Backlog is a set of features in priority order.
  
  \textbf{Development team} - is
  a set of professionals that are working on implementing features of the product. 
  The team size should be from three to eight people. The team should work only on tasks from the backlog.
  
  \textbf{Scrum master} - a person who should help the team to increase their
  productivity by enhancing the understanding of teams strengths and weaknesses.

  The main idea of scrum is that development is done in short-time periods
  called sprints. Each spring consists of several phases:
  \begin{itemize}
    \item  Sprint planning - when team decides what tasks should be done during the
  sprint .
  \item   Main phase - when actual development is done.
  \item Sprint review - when team shows the results to the product owner.
  \item Retrospective - when team discuss what can be improved.
  \end{itemize}
 
  Sprint may take from one to four weeks. The development team should decide
  what sprint length suits their needs. During sprint planning the team chooses
  which tasks will be moved from a product backlog to a sprint backlog.
  One of the restrictions is that the task in sprint backlog should be done in one sprint.
  If the team thinks that the task can not be finished in one sprint this task
  should be divided into several subtasks. 
  
  Having such one-sprint tasks helps the scrum team to keep track of the
  progress easily and gives an opportunity to receive feedback for each
  completed task at the end of the sprint. This helps to detect problems at the early stages, when
  the errors does not have a tremendous feedback.
  Even if a feature was misunderstood by the development team
  and the team  has to redone it completely , the team wastes time equal to the
  length of the spring at maximum. While in a classical waterfall model,
   a sequential design process in which progress is seen as flowing steadily through 
   the phases of all development stages, 
  the error might be found much more lately, which will have a bigger negative impact.

  Another feature of Scrum is self organization of the team. The team should
  decide by itself which tool set to use. Tasks in scrum are not assigned to
  developers by a manager, but instead developers take items from the backlog by themselves.
  This approach saves time and reduces stress, because a person can pick a task,
  which he likes and understands. Developers pick tasks that they
  can finish before the end of the sprint.

  In our case we were not developing a new product, but releasing a new version.
  We did not find any arguments to change the tools that were used in the
  previous release we describe them in section \ref{sec:toolsused} .

  \section{Test Driven Development}
  TDD is a very popular methodology of a software development. The main idea is
  to write tests first and then code. The main benefits of such approach are:
      \begin{itemize}
        \item The developer is sure that his code works as intended, because all
        his code is tested.
        \item The errors are found at early stage of the development cycle, which
        reduces the cost of fixing problems.
      \end{itemize}
      
      Three laws of TDD \cite[pp122]{cleancode}[Book page 122]
        \begin{itemize}
          \item You may not write production code until you have written a failing unit test.
          \item You may not write more of a unit test than is sufficient to fail.
          \item You may not write more production code than is sufficient to pass the currently failing test.
        \end{itemize}
    
  During the project we implemented two different type of tests - unit tests and
  integration tests. 
  
  Unit tests are used for testing individual parts of source
  code, for example method which parses a regular expression or an utility
  class, for comparing two screenshots.
  
  Integration tests combine severeal parts of the application and test 
  interaction between them. In our case integration test consist of a Web page
  running as a Vaadin application on a server and a Testbench test class. When
  Testbench test starts it opens the Webpage in the browser and operates on the
  elements on the Web page, by simulating user actions like clicking or typing
  and compare actual and expected behaviour of the elements. 
      
  \section {Tools used}
  \label{sec:toolsused}
  
  \subsection{Maven}
  Maven is a java-based software project management and comprehension tool.
  Maven is based around the central concept of a build life cycle. This means
  that the process for building and distributing a particular project(artifact) is clearly defined. There are three built-in lifecycles:
  default, clean and site. Users can define their own life cycle. 
  
  Life cycles  consist of phases.  The default life cycle includes the following phases:
  \begin{itemize}
    \item validate - validates the project is correct and all necessary
    information is available.
    \item compile - compiles the source code of the project.
    \item test - tests the compiled source code using a suitable unit testing
    framework. These tests should not require the code be packaged or deployed.
    \item package - takes the compiled code and packages it in its distributable
    format, such as JAR.
    \item integration-test - processes and deploys the package if necessary into
    an environment where integration tests can be run.
    \item verify - runs any checks to verify the package is valid and meets quality criteria
    \item install - installs the package into the local repository, for use as a
      dependency in other projects locally.
    \item  deploy - copies the final package to the remote repository for
    sharing with other developers and projects.
  \end{itemize}
  
  The life cycle phases are executed sequentially. For  example 
  running maven deploy executes all the  previous phases (validate, compile,
  test\ldots).
  
  All maven configurations are specified in the Project Object Model (POM) file.
  POM is an XML file that contains information about the project and configuration details used by Maven 
  to build the project. 

  Maven reduces the complexity of developing and maintaining big projects.
  Nowadays applications may depend on dozens of third-party libraries and
  frameworks. Managing those dependencies manually is very time consuming. Maven
  finds  and downloads the exact version of the library and adds it to the
  project.
  
   Maven profiles allow to have different configurations of your
  application for development and production or testing. All the maven
  configuration are in the same POM file, that is why editing and sharing
  configurations between members of the team is very easy. 

  Finally, you have a set of predefined configurations for your application
  for the whole team and any developer can checkout POM file from the repository
  call ``mvn deploy'' and he will have the same version of the application with all
  the specified parameters and downloaded dependencies. If you updated
  your dependencies or fixed an error, all your team members
  have to just checkout the new version of a POM file.
 
  \subsection{Trac}
  Trac is an enhanced wiki and issue tracking system for software development
  projects. Trac may include several projects. Users or developers can
  create tasks (also called tickets) for these projects. 
  
   Before development a new release a product owner goes through 
   the list of the tickets and add them to a new milestone.
  Milestone is a plan for the next release, which includes a set of tickets.

  Tickets have different value for the end user, but developers can not always
  assess that value by themselves. Product owner should help the development team
   to figure out the
  value of each ticket for the end user. Based on the value and time estimation
   each ticket should be prioritized.
  Prioritizing tickets is a very important task and should be done as soon as
  possible, preferable before coding starts. This gives a clear vision for all
  members of the team what should be done.

  In the Testbench4 project we used the Trac milestone as a product backlog. On
  the sprint planning we estimate which tasks can be completed at the end of the sprint 
  and move them to the sprint backlog. As a sprint backlog we used a scrum
  board.
  
  Scrum board is a white board, divided into several sections for example ``to be done'', ``in progress'', ``in review'',
  ``closed''. Paper stickers represent tickets and the person who is working on
  the ticket.  
  The workflow is the following - a developer picks the
  ticket from the sprint backlog queue called ``to be done'' 
  writes his name on the sticker and move it to the ``in progress'' section.
  After he submitted a patch to the code review he moves the sticker to another
  section and so on.
  
  Looking to the scrum board gives you a brief summary of every team member tasks and 
  also the current sprint progress. One can also find more detailed information
  about tickets and the project progress in Trac.
 
 \subsection{Git}
  As a version control system we used \textbf{Git} - distributed revision control system
  which focuses on speed, data integrity, and support for distributed,
  non-linear workflows. There are two types of revision control systems :
  
   \begin{itemize}
   \item Client-server - such version control systems as SVN and CVS, have a 
    a centralised model, where there's a copy of the current code on a central
    server, which users copy in order to work with locally. When a user makes
    some changes, he updates from the central version (in case other people have
    made changes in the meantime), solve conflicts (same part of code was
    changed by different people at the same time) that might have arisen, and
    then push their code back to the server. Afterwards other people can check
    it out again.

   \item Distributed revision control systems such as Git, are structured on a
    peer-to-peer basis: instead of one centralised repository. Every developer
    has their own repository and there is no main repository as in client-server
    control systems, all repositories are ``equal''. Though in practise developers
    create a ``master'' repository, where everyone push their own changes and pull
    changes made by other developers.
  \end{itemize}
  
  One of the biggest advantage of distributed systems is that repositories are
  synchronised by exchanging change-sets in the form of patches, in other words
  if you have changed two symbols, these two symbols plus some internal
  information - author, time, etc. On the contrary, in systems like SVN every
  time you pull changes from the central repository you are downloading 
  the whole snapshot of your application sources.
  
  Also Git lets developers to have their local history of changes and commits,
  but then when pushing changes to the master repository they can rebase these changes as one commit.
  This helps on one hand keep a local history of intermediate steps for
  developer, but on the other hand have only commits for completed changes
  or features in the main repository. 
  
  Git has a powerful set of tools including
  unix commands. For example to find all commits made by one person you can use
  log command and pipeline it to a pattern matching command like ``grep''.
  
   Git-blame command allows you to see the history of every line of your source
   code. If you have questions about some particular few lines of code,
   you can find an author of those lines and ask him a question.
   
   Git-bisect command - is a binary search against revision graph, which helps to find the commit which
   introduced a bug.

  \subsection{Teamcity}
  Teamcity - is a Web-based build management and continuous integration tool.
  Teamcity allows running multiple builds and tests under different platforms and environments.
  Teamcity build combines maven, ant builds, git command and bash scripts.
  
  Teamcity builds may be started automatically or manually. One option is to create a configuration 
  to run all  tests every night or to setup running tests on every git commit. Teamcity
  provides also build dependencies. If project A depends on a library B,
  Teamcity will first build library B with its dependencies and then start build
  project A.
  
  During the development cycle we used four different configurations.
  \begin{itemize}
  \item Running tests on every git commit. This configuration is started when
  Gerrit \ref{sec:gerrit} patch is submitted.
   Running all tests for all browsers is very time consuming and may take several hours. 
   That is why in this configuration includes only JUnit tests and PhantomJS
   tests, which does not need to run the actual browser.
   These tests show common errors for all browsers. Running those tests gives
    a developer a fast feedback, if his changes caused some problems.
   
   \item Running all tests on latest commit every night. This build triggers at
    specific time every night, when servers load is lower than during the day.
    This configuration includes all the heavy tests for specific browsers. All
    the tests are run on Google Chrome, Mozilla Firefox and Internet Explorer 8, 9, 10 and 11. 
    For every test suite Teamcity will run the specific browser on a test
    cluster. Running such tests is very resource consuming, but provides a
    confidence that the application is supported by all browsers.
    
    \item Snapshot build is run every night. This build publishes the latest
    version of the product to a maven repository.  Users can download
    the snapshot build with the latest version of the product, if they want to
    test new features, but do not want to wait for the release build.
    
    \item Release build is run when the team releases a new version of the
    product. This includes building all the dependencies, running all the tests,
    specifying the version of the product, creating release notes, making tag in
    the Git repository, publishing a new version to maven repository and Vaadin
    Web site.
   \end{itemize}

\subsection{Gerrit}
\label{sec:gerrit}
  Gerrit is a Web-based code collaboration tool. Gerrit allows developers to
  review patches made by other developers. Gerrit has a very easy system of evaluating patches:
  \begin{itemize}
  \item -2 (veto) - patch has major problems.
  \item -1 (disapprove) - patch has minor problems.
  \item +1 (approve) - no problems found.
  \item +2 (approve) - can be pushed to master.
  \end{itemize}
  
  The difference between +1 and +2 is that the patch can not be pushed to git
  repository without having +2. The reviewer can give +1 if he is not sure about his level of competence
  and want someone else to inspect the patch. 
  There are might be several configurations of the review process,
  figure \ref{fig:gerritTestbench} shows the process used in the
  Testbench4 project.

  Firstly, a developer submits his changes(patch) to Gerrit. Gerrit triggers the
  specific build in Teamcity. This build includes building the project and
  running tests. After this step is finished, Teamcity returns a report about the build,
  if there are problems the report is send to the developer and the patch is marked as -2. If all
  tests pass Gerrit marks the patch as ready for review and put it to the list
  of waiting for review patches.
  Afterwards the reviewer evaluates the patch. Given the patch -1 or -2 means
  that the developer should fix the problems, and submit the next version of the patch. 
  The process continues until someone  marks the patch as +2,
    meaning in can be pushed to git master repository.
    \begin{figure}
    \centering
      \includegraphics[width=0.75\textwidth]{gerrit1.png}
      \caption{Gerrit structure}
      \label{fig:gerritTestbench}
    \end{figure}
    
  Code review helps team members to follow similar code conventions, 
  keep code clean and readable and find bugs. Also code review helps developers to know more about 
  the whole project they are working in. Integrating Gerrit with an automated build tool, 
  such as TeamCity, allows to run tests before publishing commit for review. 
  The patch with failing tests is rejected automatically and an email with report
   for all failing tests sent to the author of the patch. 
   As an overall code review helps to keep source code quality on a higher level.

  \section{Architecture design}
  Testbench consist of two main modules: ``Testbench-api'' and
  ``Testbench-core''. Testbench-api contains classes and methods for
  simulating user actions on Vaadin components. Testbench-core
  handles client-server communication, screenshot comparison and
   provides mehthods for searching Vaadin components on the Web page.
   
   Testbench-api module is closely connected to Vaadin framework and changes 
   in behaviour of Vaadin components, like Button or Textfield, may require changes
   in the Testbench-api module. Testbench-api module version number matches the
   compatable Vaadin framework version. As a development team we think that
   having matching Testbench-api and framework versons will help to solve
   compatability issues, showing the users which testbench and framework
   versions they have to use.
  
  The hierarchy of classes in Testbench consists of many tens of classes and each class has tens of methods.
  Here we will describe the most important classes and the basic principles see
  figure~\ref{fig:classdiagram}.
	\begin{figure}
	\centering	
	    \includegraphics[width=0.75\textwidth] {umlDiagram.png}
	    \caption{Testbench class diagram}
	    \label{fig:classdiagram}
  \end{figure}

\textbf{TestBenchCommandExecutor} handles client server communication
and provides screenshot comparison implementation.

\textbf{AbstractHasTestBenchCommandExecutor} class provides  `\$(Class clazz)'
and `\$\$(Class clazz)' methods which create a query for searching elements of
the given type.
 `\$' method builds a recursive search query and `\$\$' a non-recursive one.
 Non-recursive search query looks only for direct children of the element, while
 it is recursive analog looks through all children of the element. Children
 elements are inner HTML-elements. In example \ref{lst:domexample}
   button and check box input elements are children elements of the div element
   with id id1.
 \lstset{style=console}
  \begin{lstlisting} [caption=Simple DOM example,label={lst:domexample}]
  <div id="id1">
	  <input type="button">
      <input type ="checkbox">
  </div>
  \end{lstlisting}
  
  
\textbf{ElementQuery<T>} used for locating Web elements(Vaadin buttons, text
fields, labels, etc.) on the Web page. Generic parameter T specifies the type of a searched element. 
 ElementQuery class provides methods for searching the element based on
 element's id,class, caption or other criteria. These methods can be considered
 as filters in the query. ElementQuery uses the Builder pattern, 
 which helps to add several filters to build a specific query and after
 the query is built execute it.

Example \ref{lst:searchButtons} shows finding all children elements of the
parent element  which are buttons:
  \lstset{style=a1listing}
\begin{lstlisting} [caption=Search for all buttons,label={lst:searchButtons}]
AbstractHasTestBenchCommandExecutor elem = getParentElement();
List<Button> allButtons=elem.$(ButtonElement.class).all();
\end{lstlisting}
  
To restrict search for buttons with caption ``ok'' we add a caption filter to
the query see example \ref{lst:searchOkButtons}.
  \lstset{style=a1listing}
  \begin{lstlisting} [caption=Search for for button  with caption "ok", label={lst:searchOkButtons}]
AbstractHasTestBenchCommandExecutor elem =  getParentElement();
List<Button> allButtons=elem.$(ButtonElement.class)
	.caption("ok").all();
  \end{lstlisting}
  
\textbf{TestBenchElemenet} - is a base class for operating Vaadin components. It
includes methods to access properties common to all Vaadin elements,
such as "getSize()'', ``getLocation()'', ``getCssValue()'', etc.
TestbenchElement class uses Selenium WebElement class as a foundation and extend
it's functionality by using JavascriptExecutor,
which allows to execute JavaScript code, and change the default element behaviour.

\textbf{TestBenchTestCase} - an abstract super class of a TestBench test.

\textbf{ParallelTest} - supports running test in parallel threads with several
browser configurations see \ref{sec:paralelTesting} for more details.

\textbf{ButtonElement, MenuBarElement, TableElement}, etc. - implement specific
Vaadin class features. The default naming conventions is Vaadin component name
plus ``Element''. In other words ButtonElement accesses buttons methods,
TableElement table methods and so on.

The important aspect is that hierarchy of testbench elements is similar to Vaadin elements.
That gives more flexibility when writing tests. The  developer/tester can
specify concrete class for getting access to specific methods of the element see example \ref{lst:specificExample}.
 
\lstset{style=a1listing}
\begin{lstlisting} [caption=Test for Vaadin table, label={lst:specificExample}]
  TableElement table= getElement();.$(TableElement.class).first();
  TableRowElement row=table.getRow(0);
 \end{lstlisting}
 
or use a more generic class to utilize method of a parent class, for example get
caption of all elements, see example \ref{lst:genericExample}.

\lstset{style=a1listing}
\begin{lstlisting} [caption=Caption test for Vaadin elements, label={lst:genericExample}]
List<TestBenchElement> elements=
	getElement().$(TestBenchElement.class).all();
List<String captions=new ArrayList<String ();
for(int i=0;i<elements.size();i++) {
	captions.add(elements.get(i).getCaption());
}
\end{lstlisting}

\section {Basic test case structure}
To use TestBench, the test case class should extend the TestBenchTestCase class,
which provides the WebDriver and ElementQuery APIs. A developer
can configure Testbench test by using following annotations:
\begin{itemize}
  \item @Rule -defines certain TestBench parameters.
  \item @Before - the annotated method is executed before each test.
  \item @Test - annotates the tested method.
  \item @After - the annotated method is executed after test.
\end{itemize}

A typical test case structure is the is the following:
\begin{itemize}
  \item Set TestBench parameters.
  \item Open the tested Web page URL.
  \item Find an element for interaction (Button, TextField).
  \item Interact with the elements (click buttons,menus,etc.).
  \item Find an element to check.
  \item Get and the value of the checked element.
  \item (optional) get screenshot.
\end{itemize}

A complete example of test UI class \ref{lst:appTestUI}  and a TestBench test
class \ref{lst:appTestClass} can be found in appendix A \ref{appendixA}.

\section {Results}
During Testbench4 development our team add new API to ease writing tests for
Vaadin applications. We added helper methos for MenuBarElement,
TreeTableElement, TwinColSelectElement,PopupDateFieldElement etc. Besides a
ParallelTest class was introduced, which supports running tests concurrently in
several threads. We have created Testbench-core and Testbench-api modules and
added separate build configurations for these modules. Overall we have completed
41 tickets and the detailed information about these tickets can be found in
table \ref{table:tickets} in appendix B \ref{appendixB}.

To improve User Experience (UX) we have tried to reduce an amount of
configuration needed for using TestBench. We provide a simple test as a
part of a build-in configuration.
Developers can use it as an example and extend it for their own requirements. Users may create a sample
Vaadin application via GUI using Vaadin Eclipse plugin or Maven build tool.
To create a sample Test via Maven user should use a Vaadin Maven
archetype see \ref{lst:mavenexample}.
\lstset{style=console}
\begin{lstlisting}[caption=Create Vaadin sample application command.,label={lst:mavenexample}]
mvn archetype:generate 
	-DarchetypeGroupId=com.vaadin
	-DarchetypeArtifactId=vaadin-archetype-application
	-DarchetypeVersion=LATEST
\end{lstlisting}

After beta release we made several usability tests.
A Vaadin developer, without experience in using TestBench, manages to create and 
run a simple ``button-click'' test in less that 15 minutes. 

Vaadin Testbench 4 is released with Commercial Vaadin Add-On License 
(CVAL). But if you want to look results of our work and try it out you have two
options:
\begin{itemize}
  \item Free 30-days trial period.
  \item One year non-commercial license. All the details how to get it are at
  Vaadin blog \cite{vaadinBlog}.
\end{itemize}

Vaadin framework is using Testbench as a main testing tool for acceptance
testing. Overall Vaadin framework has about six thousand Testbench tests which
are run nightly and also before every Vaadin release.

\chapter{Testbench vs Selenium}
\label{ch:testbenchvsselenium}
In this chapter we will summarize advantages of using TestBench against
Selenium.

\section{API built specifically for Vaadin components}
Selenium operates on the DOM of the Web page and provides only basic methods of
Web elements like ``click'' or ``sendKeys''. TestBench provides a rich API for
Vaadin components, which allows to operate on bigger parts of components, for
 example you can get a row or cell of the Table by index
 see~\ref{lst:testbench1}.
  	\lstset{language=Java}
  	\begin{lstlisting}[caption=Get Vaadin Table cell Value,label={lst:testbench1}]
TableElement table = getTableElement(); String value = table.getRow(0).getCell(1).getValue();
	\end{lstlisting}
	
\section{Client server communication}
As mentioned in section~\ref{sec:reasontestbenchdevelopment} TestBench handles 
client server syncronization.

\section{Screenshot comparesment}
Since version 4.0.0 TestBench has an API for comparing screenshots. This 
feature was introduced to help users to test UI of the application. We believe
that User Experience (UX) is very important in Web applications, and an
important part of UX is look and feel of the application.

In Web applications styling is done with CSS or SCSS. The downside of CSS 
is that changing CSS rule for one selector may affect a lot of elements on the
web page. For example change the width or margin of one element may ruin an
appearance of the whole Web page.

Manual UI testing is very difficult, because of the two main challenges:
\begin{enumerate}
  \item After some time the tester looses his concentration and does not see
  errors
  \item Small details in applications with rich UI is hard to notice for a
  human. In other words if you have several textboxes and buttons in different
  tabs or windows in the application it is hard to notice that some of them are
  not aligned.
\end{enumerate}

Both these two problems can be solved with automatic screenshot comparesment.
ImageComparesment class has an overloaded compare mehtod which takes either an
image or a path to the reference screenshot. The comparesment is done in 16x16
blocks comparing RGB values of the every pixel in this block. If there are
differences in images these parts are videleni with color, so it is easier for a
tester to find out what was the problem with the test. TestBench test can also
be configured to automatically take screenshots of the failing tests, it also
helps developers to narrow the scope of the failing test. 
tester


\section{Parallel testing}

\section{Finding UI element with mouse}

\chapter{Testbench use}
\label{ch:testbenchuse}
In this chapter we will show several examples of using TestBench for writing
automated tests and show their value for different stakeholders. 

Originally TestBench was developed as a tool for writing acceptance tests for 
Vaadin framework it also used can be used to test any application written with
Vaadin. Acceptance test determines that requirements of a specification are met.
Currently (Fall 2015) there are over 6500 tests written for Vaadin framework.
All tests are running in Chrome, Firefox and Internet Explorer 9,10,11 during
night builds and before every release. New version of Vaadin framework can
not be released even with one test failing. This strict rule helps to keep
the quality of the product on a high level.

Having automated tests allow developers to refactor code without fear of
breaking previous work. Developers may not know all the details of the framework and make mistakes,
 failing tests give sufficient information about the problem 
 and give a confidence that new changes do not break existing code.
   
 Having automated acceptance testing is extremely important for large
 open-source projects, because this reduces a cost for developers to contribute
 to a project.  All patches to the framework are reviewed by Vaadin experts and
 running tests beforehand rejects fallible code.

 While automated tests have great value, there are circumstances
 where a failing Testbench test gives a false alarm. One of the most fundamental
  problems of Web testing is that a developer can make a change that keeps
   the application completely correct, but breaks an automated test.   
That might be caused by changing DOM or CSS of the page, for example adding
an extra div may affect searching element by xPath. Such kind of problems 
occurs quite often when developing new features. Using ``screenshot on failure''
rule \ref{lst:screenshotOnFailue} helps to figure out such kind of problems. If
a developer get an error ``can not find an element on the Web page'', 
but this element is presented on a screenshot, most likely the problem is in
locating the element code.

To demonstrate the usage of Testbench we will create a test for a Vaadin table
component extension. Developing Vaadin components is outside the scope of
this work, we assume someone extended a Table component by adding a filter field to
it. Typing value in the filter field filters values of the underlying table see
figure \ref{fig:filtertable}.
	\begin{figure}
	\centering
	\includegraphics[width=0.75\textwidth]{filtertable}
	\caption{Table component extension example.}
	\label{fig:filtertable}
	\end{figure}

An essential  part of a test for the filter feature is represented in listing
\ref{lst:testbenchTest}. 

 \lstset{style=a1listing}
  \begin{lstlisting} [caption=Example of table test,label={lst:testbenchTest}]
TableElement table = getTableElement();
TextFieldElement filterElement=table.
	$(TextFieldElement.class).id("filter-field").first();
filterElement.setValue("special");

//Comparing filtered values
TableRowElement row=table.getRow(1);
assertEquals(row.getCell(0).getValue(),"special");
  \end{lstlisting}

In addition to having value throughout the development life cycle,
Testbench tests are valuable artifacts to get end-users feedback.
Because Testbench tests are executed in a browser, tests can be used 
for demonstrating framework or application features to an end-user. The key
downside of this approach, that you can not demonstrate the behaviour of the
application till it is written already. In a worst case scenario, a developer
may spend time on implementing features that do not fit user requirements and
have to make tremendous changes after demonstrating the application to the end
user. Using Behaviour Driven Development(BDD) tehnique helps to improve this
situation.

Behaviour Driven Development is a software development process which combines
TDD and domain driven design. The key to success of BDD is the executable
acceptance tests that describe the expected behavior of a feature and its
acceptance criteria in the form of scenarios using simple and
business people readable syntax\cite{bddArticle}. BDD uses a business readable
language to describe software's behavior while hiding its implementation
details.

During a BDD process a business user works with a business analyst
to identify business requirements. These requirements are expressed as a story
using the following template:

\begin{description}
  \item[Given] an initial context.
  \item[When] an event occurs,
  \item[Then] get some outcomes. 
\end{description}

Business and technology should refer to the same system in the same way, that is
why BDD relies on the use of a very specific and small vocabulary to minimise miscommunication \cite{bddWebSite}.
Because a developer and an end-user are using same language, it is much easier
to negotiate about application requirements and as a result make developing
process faster and more agile.

In the next section we will show how to improve this test by using
Behaviour-Driven Development(BDD) framework.

\section{Integrating with Behaviour-Driven Development frameworks}

The main goal of BDD is to get executable specifications of a system.
In other words BDD frameworks allow  to write user-stories in
common language, for example English, and associate them with automated
acceptance tests.

Testbench can be integrated with such BDD frameworks as JBehave
\cite{jBehaveSite} or Cucumber\cite{cucumberSite}.
Tests in JBehave are called scenarios see example of JBehave scenario
\ref{lst:scenario}.

 \lstset{style=console}
  \begin{lstlisting} [caption=JBehave scenario,label={lst:scenario}]
Scenario: filter table contents
Given web-page with table
When typing special to filter field
Then value in row 1 and cell 0 is special
  \end{lstlisting}
User story steps are matched into actual Java tests using annotations. 
The method with an annotation interact with an application and perform the actions needed.
Since TestBench tests are pure Java code and BDD steps can be run as JUnit tests, we can combine these to make
JBehave run TestBench tests.

We start by extending the TestBenchTestCase  and use JBehave's ``\@BeforeScenario''
annotation to open a tested Web page.  @Given @When and @Then annotations are
linked with corresponding steps in the user scenario.
To pass parameters from the user-scenario step to a Java method "\$"- special symbol is used.
JBehave implicitly casts passed value to a parameters type. The complete example
of the test is in appendix A \ref{lst:testbenchTest}.

To link a Java class and a textual story file we need to create a configuration class.
The simplest configuration is a one-to-one mapping between a Java class and a textual story file
see example in appendix A \ref{lst:jbehaveConfig}. 

So we can have both a user-story for our test explaining what should be done and a browser executed
test showing the actual implementation. Picture with user story and browser implementation. We believe that user-stories 
scenarios greatly ease communication between stakeholders, 
especially if some of them do not have relevant technical background.
So user-stories can be shared between stakeholders to show what is done and what
is planning to be done, if there are questions 
executing these user-stories in a browser will help to reveal more details about it. 
  

\iffalse
 \subsection {Comparing Testbench with other tools}
    Here I will compare testbench to some other testing tools for example
    Selenium, what are benefits of using Testbench. The main idea is to focus
    that Vaadin is a client-server framework, so all Vaadin components have
    client and server side code, and it's important to test both. I want to show
    what problems may happen if using only Selenium and test only client side
    code, so that server side isn't tested. 
    
    Also I would like to compare how complicated is to right Selenium and
    Testbench tests. How much code is needed to test button click for example,
    or sending text to a textField.
    
    Compare speed of running tests. In Vaadin we run tests every night, and
    sometimes it takes too much time, so I would like to mention ,what are the
    problems/challenges in having a lot of tests.

      
 \subsection {Vaadin integration}
    Testbench using some features of Vaadin framework, for example searching
    elements by vaadin selectors. So I want to describe how integration with
    Vaadin is done. What are the challenges. For example what would happen when
    using different versions of Testbench with different versions of Vaadin.
\fi    

 \section {Discussion and Conclussion}
 	Middle of April/ End of April
 	\subsection {Summary}
 	  What has been done. What were the challenges how they were solved.
 	\subsection {Advantages and disadvantages}
 	\subsection {Future work}
 	\nocite{*}
 	\bibliographystyle{unsrt}
  \bibliography{masterthesis} 
  	
\appendix
\section{Appendix A}
\label{appendixA}

\lstset{style=a1listing}
\begin{lstlisting} [caption=Test Web Page class,label={lst:vaadintest}]
public class TestWebPageClass {
  static final String ASSERT_ELEM_ID="assertElementId";
  static Map<AbstractElement,String> map = new HashMap();
  static  {
         map.put(new TextField(),"textField");
         map.put(new ComboBox(), "combobox");
  }
  
  TextField assertionElement=new TextField();
  public void createTestWebPage () {
    Iterator it = classToAssertValue.entrySet().iterator();
    while (it.hasNext()) {
     it.getKey().setId(it.getValue());
     addElementToWebPage(it.getKey());
     it.getKey().setValueChangeListener(event-> {
       assertionElement.setValue(it.getValue());
     });
    }
    
    addAssertElement();
  }  
    
    public static <AbstractElement,String> getMap() {
      return map;
    }
}
\end{lstlisting}

\lstset{style=a1listing}      
\begin{lstlisting} [caption=Test class,label={lst:vaadintest2}]
public class ValueChangeListenerTestClass {
  <AbstractElement,String> map=TestWebPageClass.getMap();
  String assertElementId=TestWebPageClass.ASSERT_ELEM_ID;
  UIElement assertElement=findElementById(assertElementId);
       
  @Test
  public void testValueChangeListener() {
    openWebPage();
    Iterator it = map.entrySet().iterator();
          
    while (it.hasNext()) {
      Map.Entry pair = (Map.Entry)it.next();
      UIElement elem=findElementById(map.getValue());
      elem.setValue(``foo'');
      String assertMessage=``Element with id=''+pair.getValue()
          + ``has wrong value'';
        
      Assert.assertEquals(assertMessage,assertElement.getValue(),
          pair.getValue);
      }
    }
}
\end{lstlisting}




\lstset{style=console}
\begin{lstlisting} [caption=Generated C\&R test  example using Selenium IDE ,label={lst:capturereplay}]
<tr> 
  <td>open</td>
  <td>
    /tutorials/selenium/selenium_record_replay.htm
  </td>
  <td>type</td>
  <td>id=number1</td>
  <td>123</td>
</tr>
<tr>
  <td>type</td>
  <td>id=number2</td>
  <td>123</td>
</tr>
<tr>
  <td>click</td>
  <td>id=add</td>
  <td></td>
</tr>
<tr>
  <td>verifyValue</td>
  <td>id=total</td>
  <td>246</td>
</tr>
\end{lstlisting}
	 
  \lstset{style=a1listing}
  \begin{lstlisting} [caption=Selenium test example,label={lst:starthub2},language=java]
public  class TestExample throws MalformedURLException {
  WebDriver driver;
   String UIUrl,nodeURL;
   @BeforeTest public void setUp() {
    UIUrl="http://app.example.com/hellopage"; 
    hubURL="http://192.168.1.2:5566/wd/hub";
    DesiredCapabilities capability=DesiredCapabilities.firefox();
    driver=new RemoteWebDriver(new URL(hubURL),capability);
  }
    
  @Test
  public void test1() {
    driver.get(UIUrl);
    Assert.assertEquals("Welcome",driver.getTitle());
  }
  
  @AfterTest
  public void afterTest() {
    driver.quit();
  }
}
  \end{lstlisting}	 
	 
\lstset{style=a1listing}
\begin{lstlisting} [caption=Test UI class example,label={lst:appTestUI}]
@Theme("mytheme")
@Widgetset("com.example.testbench.MyAppWidgetset")
public class MyUI extends UI {
	@Override
	protected void init(VaadinRequest vaadinRequest) {
	final VerticalLayout layout = new VerticalLayout();
	layout.setMargin(true);
	setContent(layout);

	Button button = new Button("Click Me");
	button.addClickListener(new Button.ClickListener() {
	@Override
	public void buttonClick(ClickEvent event) 
		layout.addComponent(new Label("Thank you for
			clicking")); }
	});
	layout.addComponent(button);

	}
    
	@WebServlet(urlPatterns = "/*", name = "MyUIServlet",
		asyncSupported = true)
	@VaadinServletConfiguration(ui = MyUI.class,
		productionMode = false)
	public static class MyUIServlet extends VaadinServlet {
	}
}
\end{lstlisting}

\lstset{style=a1listing}
\begin{lstlisting} [caption=Test Bench class example,label={lst:appTestClass}]
public class ButtonTest extends TestBenchTestCase {
	public static final String baseUrl =
		"http://localhost:8080";
    
	@Before
	public void setUp() throws Exception {
		// Set WebDriver
	setDriver(new FirefoxDriver());
}

	@After
	public void tearDown() throws Exception {
		getDriver().quit();
	}
    
	@Test
	public void testClick() {
	//Open URL
	getDriver().get(baseUrl + "?restartApplication");
	ButtonElement button = $(ButtonElement.class).first();
		button.click();
		LabelElement label = $(LabelElement.class).first();
		String text = label.getText();
		Assert.assertEquals("Thank you for clicking", text);
	}
    }
\end{lstlisting}


  	\lstset{style=a1listing}
  	\begin{lstlisting} [caption=Selenium test for Vaadin application,label={lst:seleniumVaadin}]
public class AppTest extends TestCase{
  WebDriver driver;
  String UIUrl, nodeURL;	
	@Override @Before
    public void setUp() {
      UIUrl = "http://demo.vaadin.com/dashboard/";
      driver = new FirefoxDriver();
    }

  @Test
  public void testWithoutTestbench() {
    driver.get(UIUrl);
    driver.manage().timeouts().
      implicitlyWait(5, TimeUnit.SECONDS);
    List<WebElement> elements = driver.findElements(By
      .className("v-button"));
    if (elements.isEmpty()) {
      throw new RuntimeException("No buttons found");
    }
    elements.get(0).click();
    driver.findElement(By.id("dashboard-edit")).click();
    WebElement searchField =
     driver.findElements(By.className("v-textfield")).get(0);
    
    searchField.clear();
    searchField.sendKeys("New Dashboard");
    searchField.sendKeys(Keys.TAB);
    WebElement searchButton = findButtonByCaption("Save");
    searchButton.click();
    driver.manage().timeouts().
      implicitlyWait(5, TimeUnit.SECONDS);
    String title = driver.findElement(By.
      id("dashboard-title")).getText();
    Assert.assertEquals("New Dashboard", title);
    }

  public WebElement findButtonByCaption(String caption) {
    List<WebElement> buttons = driver
      .findElements(By.className("v-button"));
    for (WebElement button : buttons) {
      if (button.getText().equals(caption)) {
        return button;
      }
    }
    return null;
    }

  public WebElement findButtonByCaption(
    WebElement parent, String caption) {
   
    List<WebElement> buttons = parent
      .findElements(By.className("v-button"));
    
    for (WebElement button : buttons) {
      if (button.getText().equals(caption)) {
        return button;
      }

    }
    return null;
  }

 @After
 public void afterTest() {
  driver.quit();
 }
}
\end{lstlisting}

	
	\ref{lst:testbenchVaadin}.
	\lstset{style=a1listing}
  	\begin{lstlisting} [caption=TestBench test,label={lst:testbenchVaadin}]
public class TestBenchTest extends TestBenchTestCase {

    WebDriver driver;
    String UIUrl, nodeURL;

    @Before
    public void setUp() throws Exception {
        UIUrl = "http://demo.vaadin.com/dashboard/";
        setDriver(new FirefoxDriver());
    }

    @Test
    public void test1() {
        getDriver().get(UIUrl);
        $(ButtonElement.class).first().click();
        $(ButtonElement.class).id("dashboard-edit").click();
        TextFieldElement searchField =
           $(TextFieldElement.class).first();
        searchField.setValue("New dashboard");
        $(ButtonElement.class).caption("Save").first().click();

        String title = $(LabelElement.class).
          id("dashboard-title").getText();
        Assert.assertEquals("New dashboard", title);
    }

    @After
    public void afterTest() {
        // driver.quit();
    }
}  	
\end{lstlisting}

 \lstset{style=a1listing}
\begin{lstlisting} [caption=JBehave test example,label={lst:jbehaveTest}]
public class FilterTableSteps extends TestBenchTestCase {
	TableTestUI page;
	@BeforeScenario
		//open web page
	public void beforeScenario() {
		setDriver(TestBench.createDriver(
			new FirefoxDriver()));
		getDriver().get("http://localhost:8080");
	}
	
	@AfterScenario
	public void afterScenario() {
		getDriver().quit();
	}
	
	@Given("web-page with table")
		public void theFrontPage() throws Throwable {
		page = PageFactory.initElements(
			getDriver(), TableTestUI.class);
	}
	
	@When("typing $value to filter field")
	public void filterTable(String value)
		throws Throwable {
		TableElement table =
			page.$(TableElement.class).first();
		TextFieldElement filterElement=
			table.$(TextFieldElement.class).
			id("filter-field").first();
		filterElement.setValue("special");
	}
	
	
	@Then("value in row $rowNumber 
		and cell $cellNumber is $expectedValue")
	public void checkValueInCell(int rowNumber,
		int cellNumber,String expectedValue) throws Throwable {
		TableElement table = page.$(TableElement.class).first();
		TableRowElement row=table.getRow(rowNumber);
		assertEquals(row.getCell(cellNumber).
			getValue(),expectedValue);
	}
}
\end{lstlisting}

\lstset{style=a1listing}
\begin{lstlisting} [caption=JBehave configuration example,label={lst:jbehaveConfig}]
public class SimpleConfig extends JUnitStory {
	//Specify the configuration, starting from default
	//MostUsefulConfiguration, // and changing only what is needed
    
	@Override
	public Configuration configuration() {
		return new MostUsefulConfiguration()
		// where to find the stories
		.useStoryLoader(new LoadFromClasspath(this.getClass())) 
		// CONSOLE and TXT reporting
		.useStoryReporterBuilder(new StoryReporterBuilder()
			.withDefaultFormats()
			.withFormats(Format.CONSOLE, Format.TXT));
	}
 
	// Specify the steps classes
	@Override
	public InjectableStepsFactory stepsFactory() {    	
	// varargs, can have more than one step class
		return new InstanceStepsFactory(configuration(),
		new FilterTableSteps());
	}
}
\end{lstlisting}

  	
\appendix
\section{Appendix B}
\label{appendixB}
\begin{table}
	\begin{tabular}{|l|l|}
	\hline
	Ticket number & Description \\ \hline
	\hline
	15092 & Add getValue/SetValue for PopupDateFieldElement. \\ \hline
	14405 & Remove getValue/SetValue from AbstractFieldElement class. \\ \hline
	15088 & Change API for TableElement. \\ \hline
	15097 & Remove the Getting Started PDF. \\ \hline
	15102 & Fix the driver instantiation before license check issue. \\ \hline
	15032 & Changes to the API of MenuBarElement. \\ \hline
	15091 & Change BrowserUtil class.\\ \hline
	15089 & Accordion/TabSheetElement now throw NoSuchElementException.\\ \hline
	15085 & Remove getElementInCell from TableRowElement class.\\ \hline
	14921 & Update the JavaDoc of BrowserConfiguration.\\ \hline
	15086 & Change api for TwinColSelectElement.\\ \hline
	15087 & Change api for NotificationElement \\ \hline
	15093 & Modify API for NativeSelectElement.\\ \hline
	
	14918 & Add setValue method to OptionGroupElement.\\ \hline
	14919 & Make TwinColSelectElement.init() protected.\\ \hline
	14920 & MenuBarElement now checks for Vaadin version 7.3.4.\\ \hline
	14438 & Update JavaDoc for scroll/scrollLeft).\\ \hline
	14163 & Fix compare screen javadoc.\\ \hline
	13606 & Replace licensing.txt with license.html \\ \hline
	14403 & Add TableHeaderElement class .\\ \hline
	14875 & Change TextFieldElement.setValue() to send Keys.TAB.\\ \hline
	13773 & Add getRow() and toggleExpanded() for TreeTableElement.\\ \hline
	14516 & Fix nativeSelect setValue and selectByText tests for phantomJS.\\ \hline
	14385 & Add contextClick for TableElement.\\ \hline
	14889 & Fix the standalone package to contain all classes\\ \hline
	14434 & TabSheetElement now works with tabs without a caption .\\ \hline
	14426 & Fix javadoc for first() and get() in ElementQuery class .\\ \hline
	13770 & Add getRow() for TableElement.\\ \hline
	14384 & Add contextClick() and doubleClick() for TestBenchEleement.\\ \hline
	14356 & Fix getText() method of NotificationElement.\\ \hline
	14486 & Fix selectByText() for ComboBoxElement.\\ \hline
	14313 & Change notification close element.\\ \hline
	14778 & Update the PhantomJS Driver dependency.\\ \hline
	14068 & Add readOnly method to all vaadinElement classes()\\ \hline
	14808 & Exists() now returns false when the search fails.\\ \hline
	13826 & Fix scroll() and scrollLeft() for TableElement.\\ \hline
	14819 & Add scroll and scrollLeft for PanelElement.\\ \hline
	13768 & Add API to notification element\\ \hline
	13769 & Add getHandler to SliderElement\\ \hline
	14372 & Fix get popup suggestions in a ComboBoxElement\\ \hline
	14790 & @BrowserConfiguration methods must no longer be static\\ \hline
	\end{tabular}
	\caption{List of closed tickets.}
	\label{table:tickets}
\end{table}
\end{document}

//TODO
{Other frameworks ideas} part give a name to all aprroaches or at least a link
add picture of the framework tool